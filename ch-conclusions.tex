\chapter{Summary and Outlook}
\label{ch:Concl}
In this chapter we summarise our result and conclusion and provide an outlook for further work.
We compare Bayesian model to the regularisation approach and elaborate on the differences between sampling based methods and the TT approximation.
Lastly, we discuss what our results mean in the physical context of atmospheric modelling and for future deployment of an atmospheric sounder.


\section{Regularisation Solution vs. Hierarchical Bayesian Approach}
To obtain one solution of this inverse problem using a regularisation approach we need 200 of $\bm{x}_{\lambda}$.
Comparing that to the sample based approach we need to draw 1000 samples from the marginal posterior or 400 function evaluations of $\pi(\lambda, \gamma | \bm{x})$.
Then we need 20  evaluations of $\bm{x}_{\lambda}$ and $\bm{B}^{-1}_{\lambda}$ to characterise the full posterior.
Either method takes a few seconds but we obtain a posterior distribution, from which we can generate multiple feasible solutions to the inverse problem instead of only one regularised solution.
Using a Bayesian framework provides true errors, which regularised methods can not.


\section{Sampling Methods vs. TT Approximation}

Using the TT approximation we need far fewer function evaluations compared to sample based methods of the target distribution.
For the 2D marginal the TT is similarly fast.

But the TT approximation needs less function evaluations than the MWG sampler= $400$ to $10000$
Once we obtain a good TT approximation we can either use the marginal of that distribution for quadrature or draw samples at the cost of 2 function evaluations per independent sample (MH correction step).
Even thought there may be more efficient samplers than the \texttt{t-walk}.
In high dimension we need roughly
A good comparsin may be 1000 sample from the t-walk takes 2100000 function evaluation with a maximum IACT of 500.
In comparison we need  24120 and then a 1000 samples take 2000 ficntion evalution.

The t-walk is a robust easy to implement sampling method, of course one could employ a more efficient sampler such as a gibbs sampler or something similar \cite{}.
Within the TT approximation we can run into numerical problems and we have to predeinfe the grid.

Reduce correlation structure by rotating coordinate system



One way of solving this issue could be to use a different basis set such as Lagrange polynomials as these exactly fit to a Gaussian or Chebychev polynomial as basis functions.
Another idea is to use different reference measure for integration, such as a Gaussian measure instead of the current Lebesgue measure.
Or that the TT finds normalisation constants automatically.


Low rank bound as in \cite{Rohrbach2022tterror}

%\subsection{Intuition of TT}
%more correlation hiegher ranks and/or grid point few sweeps and order is important to correlation structure 
%keep ranks as low as possible and increse number of sweeps
%less correlation fewer rank and gridpoints maybe more cheaper sweeps
\subsection{Approximations Errors}
We consider the approximation errors of the functions $f(\lambda)$, $g(\lambda)$ and propagation error into the marginal posterior for sampling of about $7\%$ and $1\%$ negligible.
We not the depending on the problem different approximations for $g$ and $f$ may be suitable.
The TT approximation error from the marginal posterior is with about $10\%$ also good enough since we do not believe that our model is accurate enough to capture those differences.
When approximation the affine map we get an relative error of about $0.4\%$, which is much smaller than the relative difference in between noise free and noisy data of approximately of $1 \%$.
The posterior temperature profiles is similar to the prior profiles, as also seen in marginal posterior Fig. \ref{fig:PostHistTT0} to \ref{fig:PostHistTT4}.


\section{Atmospheric Physics}
Here we want to say how informative the data is and what we can about the ozone pressure and temperature profiles.

So all the samples as in Fig. \ref{fig:O3SolplsReg} and Fig. \ref{fig:O3Samp}, present valid solutions to the inverse problem.
Hence, we can see that the variability of ozone in the upper atmosphere is large and that we do not capture the ozone peak around $80$km.
Measuring in noisy regions does not improve the results rather investing in a better measuring device with higher SNR is better.
Better modelling parametrise the Ozone reaction taking place in the atmosphere. or use truncated/ positive priors

Highly pressure correltated to ozoene.
Temperature mostly uninformative.
We can already see that in the prior analyse, as the pressure temperature ratio does inherit the exponential structure of the pressure profile.
So the posterior pressure profile is much more informative, see marginal for $b$ in Fig. \ref{fig:PostHistTT4}. 
So we can retrieve an informative pressure profile for the pressure but not for temperature.


We could fix that by choosing a more restrictive priors for $b$ in pressure but that would not be objective 
Instead we should really work on a better mode for ozone.
We can get rid of the pressure skew at a SNR of $\approx 10000$

Then we can include more measurement specific details such as the pointing accuracy.


