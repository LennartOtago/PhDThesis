\chapter{Conclusions}
\label{ch:Concl}
In this chapter we draw conclusion based on the results from the previous chapter.
We compare the regularised solution to the mean and to the samples from the full posterior.
We elaborate on the occurring approximation errors.
We compare the marginal posterior distributions based on the drawn samples and from the TT-decomposition.
While elaboration about the different methods, we also elaborate on how informative the data and what the means in terms of ozone, pressure and temperature profile. 


\section{Regularisation vs MTC}
As already mentioned the regularisation approach only provides on solution, see Fig. \ref{fig:O3SolplsReg}.
In comparison we can provide the mean and variance of the full conditional posterior distribution, as well as the sample mean of the truncated full conditional posterior distribution.
So all the samples as in Fig. \ref{fig:O3SolplsReg} and Fig. \ref{fig:O3Samp}, present valid solutions to the inverse problem.
Hence, we can see that the variability of ozone in the upper atmosphere is large and that we also not capture the peak around of the ground truth ozone profile.
\begin{itemize}
	\item distribution vs one solution
	\item hierarchial model model lambda
	\item This maximises the full conditional distribution for $\bm{x}|\bm{y}$, so is not, as often erroneously stated, the maximum a posteriori (MAP) estimate which includes at the hyper-parameters.
	\item Sol Reg vs Mean, vs samples vs sample mean, see L-Curv Fig
\end{itemize}

\section{Approximation Errors}

We can approximate the functions $f(\lambda)$ and $g(\lambda)$ sufficiently well, see sec. \ref{}.
Due to the noise we have relative error of approximately of $1.7 \%$ which is much larger than the error due to the affine approximation, which $0.4 \%$.
The error linear to non-linear.

The error due to the TT approximation for the marginal posterior is about $10\%$.
The mean error of the TT approximation for the pressure temperature ratio posterior is about $15\%$.
We consider that accurate enough since we do not believe that our model is accurate enough to capture those differences.

\begin{itemize}
	\item  we can approximate the non-linear forward model well within the relative difference between the noisy data and noise free non-linear data, which is approximately $ 1.7 \%$.
\end{itemize}

\section{Sampling vs TT}
We can conclude that the TT approximation is faster or as fast as sampling methods.
For the marginal posterior $\pi(\gamma, \lambda | \bm{y})$ the calculation of the TT-cores takes $0.1$s, which we consider similar to the sampling time of $0.5$s.
But the TT approximation needs less function evaluations than the MWG sampler.
More precisely, the TT needs $n_{\text{tot}} = 2n_{\text{sweeps}}((d-2)r^2n+ 2nr) $ function evaluations, which is in this case with number of sweeps $n_{\text{sweeps}} =2$ and rank $r=10$
\begin{itemize}
	\item time
	\item number of function evaluations
	\item numerical limits (python package)
	\item errors MTC Sampling vs real
	\item TT vs Real Marg Post and Other post
\end{itemize}



\chapter{Summary and Outlook}
\label{ch:SumOut}
\section{Atmospheric Physics}
\begin{itemize}
	\item how informative is the data
	\item Data sensitive informative uninformative, Ozone in higher altitude, pressure  temp \cite{}
\end{itemize}

\section{Measurement Device}
\begin{itemize}
	\item Data sensitive informative uninformative, Ozone in higher altitude, pressure , temperature
	\item SNR v s Pointing accuracy from experience
	\item include pointing accuracy, weighted mean for pointing accuracy
	\item nadir geometry for higher altitudes citation
\end{itemize}

\section{Methods}
\begin{itemize}
	\item graph Laplacian
	\item calculating the covariance can be expensive and if that is the case the RTO methods is the preferred choice.
	\item Through exploratory analysis we found that instead of increasing the ranks optimising the tensors by sweeping over them gives better approximations and is faster, which is crucial in higher dimensions as in section
	\item TT other bases
	\item speed gridsize intital ranks
	\item Machine learning or other methods for affine map
	\item regularised vs posterior
	\item truncated mean
\end{itemize}

The TT alforith with has nkber of funciton evaluaation set with constant rank r 
$( (D-2) r \times n\times r + 2  \times n \times r)2 \times n_{sweep}$
400 for TT marg