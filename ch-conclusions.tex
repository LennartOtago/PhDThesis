\chapter{Conclusions}
\label{ch:Concl}

\section{Methods}

\subsection{Regularisation vs MTC}
\begin{itemize}
	\item distribution vs one solution
	\item hierarchial model model lambda
	\item This maximises the full conditional distribution for $\bm{x}|\bm{y}$, so is not, as often erroneously stated, the maximum a posteriori (MAP) estimate which includes at the hyper-parameters.
	\item Sol Reg vs Mean, vs samples, see L-Curv Fig
\end{itemize}

\subsection{Sampling vs TT}
\begin{itemize}
	\item time
	\item number of function evaluations
	\item numerical limits (python package)
\end{itemize}

\section{Atmospheric Physics}
\begin{itemize}
	\item Data sensitive informative uninformative, Ozone in higher altitude, pressure  temp
	\item truncated mean
\end{itemize}


\chapter{Summary and Outlook}
\label{ch:SumOut}

\section{Atmospheric Physics}
\begin{itemize}
	\item Data sensitive informative uninformative, Ozone in higher altitude, pressure , temperature
	\item SNR v s Pointing accuracy from experience
	\item include pointing accuracy, weighted mean for pointing accuracy
	\item nadir geometry for higher altitudes citation
\end{itemize}

\section{Methods}
\begin{itemize}
	\item graph Laplacian
	\item calculating the covariance can be expensive and if that is the case the RTO methods is the preferred choice.
	\item Through exploratory analysis we found that instead of increasing the ranks optimising the tensors by sweeping over them gives better approximations and is faster, which is crucial in higher dimensions as in section
	\item TT other bases
	\item speed gridsize intital ranks
	\item Machine learning or other methods for affine map
	\item regularised vs posterior
\end{itemize}

The TT alforith with has nkber of funciton evaluaation set with constant rank r 
$( (D-2) r \times n\times r + 2  \times n \times r)2 \times n_{sweep}$
400 for TT marg