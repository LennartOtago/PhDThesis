\chapter{Summary and Outlook}
\label{ch:Concl}
In this chapter we draw conclusion based on the results from the previous chapter.
We compare the regularised solution to the mean and to the samples from the full posterior.
We elaborate on the occurring approximation errors.
We compare the marginal posterior distributions based on the drawn samples and from the TT-decomposition.
While elaboration about the different methods, we also elaborate on how informative the data and what the means in terms of ozone, pressure and temperature profile. 


\section{Regularisation vs MTC}
As already mentioned the regularisation approach only provides on solution, see Fig. \ref{fig:O3SolplsReg}.
In Fig. \ref{fig:LCurve} we plot samples from the full conditional, which lay above L-Curve and make sense in terms of the Lagrange multipliers as the point on the L-Curve can be seen as extreme values.
So the regularised estimate does not correlate to posterior solutions of the inverse problem.
We note that the mean of full conditional is very similar to the regularised solution but is also some sort of an extreme value.

In comparison to the regularisation solution we can provide the mean and variance of the full conditional posterior distribution, as well as the sample mean.



\section{Sampling vs TT}
We can conclude that the TT approximation is faster or as fast as sampling methods.
For the marginal posterior $\pi(\gamma, \lambda | \bm{y})$ the calculation of the TT-cores takes $0.1$s, which we consider similar to the sampling time of $0.5$s.
But the TT approximation needs less function evaluations than the MWG sampler.
More precisely, the TT needs $n_{\text{tot}} = 2n_{\text{sweeps}}((d-2)r^2n+ 2nr) = 400$ function evaluations, with number of sweeps $n_{\text{sweeps}} =2$ and rank $r=10$ and grid size $n = 25$ compared to $10000$ samples.

When approximating the posterior distribution of the temperature pressure ratio we are much faster compared to sampling methods.
Since the parameter space is $16$-dimensional we have to run the t-walk for about $2$ million steps.
In addition of checking the trace of the samples, we also estimate the IATC with \cite{UwerrM} see Tab. \ref{tab:priors}.
Since for shorter chains with a sample size of $10^6$ the error for the IATC estimate is much larger we decide to a sample size of $4 \cdot 10^6$ is sufficient.
This comes with a sampling time of $20$mins, much larger compare to the $2.5$min.
Which makes sense as we need $n_{\text{tot}} = 2n_{\text{sweeps}}((d-2)r^2n+ 2nr) = 384838438$ function evaluations.
We also note that we do run into problems especially in higher dimensional functions as we have a large range of values and hence introduce the constant c as already mentioned.
The t-walk is more robust but the TT approximation is faster.

But both the samples and the TT approximation point towards the same results.

Reduce correlation structure by rotating coordinate system
%\subsection{Intuition of TT}
%more correlation hiegher ranks and/or grid point few sweeps and order is important to correlation structure 
%keep ranks as low as possible and increse number of sweeps
%less correlation fewer rank and gridpoints maybe more cheaper sweeps

\section{Approximation Errors}
We consider the approximation errors of the functions $f(\lambda)$, $g(\lambda)$ and propagation error into the marginal posterior for sampling of about $10\%$ good enough.
The TT approximation error from the marginal posterior is with about $10\%$ also good enough since we do not believe that our model is accurate enough to capture those differences.

When approximation the affine map we get an relative error of about $0.4\%$, which is much smaller than the relative difference in between noise free and noisy data of approximately of $1.7 \%$.
We like to note that the relative difference 
The error linear to non-linear.
Low rank bound as in \cite{Rohrbach2022tterror}
\section{Atmospheric Physics}
Here we want to say how informative the data is and what we can about the ozone pressure and temperature profiles.

So all the samples as in Fig. \ref{fig:O3SolplsReg} and Fig. \ref{fig:O3Samp}, present valid solutions to the inverse problem.
Hence, we can see that the variability of ozone in the upper atmosphere is large and that we do not capture the ozone peak around $80$km.
The posterior temperature profiles is similar to the prior profiles, as also seen in marginal posterior Fig. \ref{fig:PostHistTT0} to \ref{fig:PostHistTT4}.
We can already see that in the prior analys, as the pressure temperature ratio does inherit the exponential structure of the pressure profile.
So the posterior pressure profile is much more informative, see marginal for $b$ in Fig. \ref{fig:PostHistTT4}. 
So we can retrieve an informative pressure profile for the pressure but not for temperature.

Ideally we should do this iteratively update ozone and then temperature and pressure until proven convergence.
\chapter{Outlook}
\label{ch:SumOut}

\section{Measurement Device}
%From the experience we can say that pointing accuracy is more important than the signal-to-noise ratio.
Then we can include more measurement specific details such as the pointing accuracy.
Then we could sample measurement $N_{\Gamma}$ geometries $\Gamma^{(k)}\sim \pi(\Gamma)$ so that the posterior $\pi(\bm{x}|\bm{y}) \approx 1/N_{\Gamma} \sum_{\Gamma} \pi(\bm{x}, \Gamma^{(k)}|\bm{y})$ and include other measurement device specific parameters.

\section{Methods}
Here we point out possibilities for improvement of the currently used methods
\subsection{TT approximation}
Within the TT approximation we run into numerical problems.
One way of solving this issue could be to use a different basis set such as Lagrange polynomials as these exactly fit to a Gaussian or Chebychev polynomial as basis functions.
Another idea is to use different reference measure for integration, such as a Gaussian measure instead of the current Lebesgue measure.
Or that the TT finds normalisation constants automatically.

\subsection{Sampling}
The t-walk is a robust easy to implement sampling method, of course one could employ a more efficient sampler such as a gibbs sampler or something similar \cite{}.

\subsection{Model}
Since we have to truncate the full conditional at the end the model is not accurate enough to eliminate those values.
This was to show that we can to a more comprehensive analysis compared to a regularised method.
Ideally we like to use a more accurate model where we parametrise ozone, similar to the pressure and temperature profile.
In doing so one would has to know much more about ozone in different altitudes.
Then we possibly could employ a different graph Laplacian based on a different structure of ozone.
And when we approximate the non-linear forward map with a affine map using a linear solver we could of course use other methods such as the machine learning methods.

