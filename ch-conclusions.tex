\chapter{Summary and Outlook}
\label{ch:Concl}

\section{Atmospheric Physics}
\begin{itemize}
	\item Data sensitive informative uninformative, Ozone in higher altitude, pressure , temperature
	\item SNR v s Pointing accuracy from experience
	\item include pointing accuracy, weighted mean for pointing accuracy
\end{itemize}

\section{Methods}
\begin{itemize}
	\item graph Laplacian
	\item calculating the covariance can be expensive and if that is the case the RTO methods is the preferred choice.
	\item Through exploratory analysis we found that instead of increasing the ranks optimising the tensors by sweeping over them gives better approximations and is faster, which is crucial in higher dimensions as in section
	\item TT other bases
	\item speed gridsize intital ranks
	\item Machine learning or other methods for affine map
	\item regularised vs posterior
\end{itemize}