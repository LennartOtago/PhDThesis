\chapter{Affine Approximation of the Non-Linear Model}
\label{ch:affine}

\newcommand*{\vertbar}{\rule[-1ex]{0.5pt}{2.5ex}}
\newcommand*{\horzbar}{\rule[.5ex]{2.5ex}{0.5pt}}
The forward map, introduced in Chapter~\ref{ch:formodel}, poses a weakly non-linear forward problem.
One could tackle this non-linear inverse problem by fixing the absorption at a previously obtained parameter state and treating this as a linear inverse problem.
After each parameter sample the absorption is then iteratively updated.
Instead, as in Fig.~\ref{fig:affinStrat} illustrated, we approximate the non-linear model using an affine map, which is a linear map with a translation e.g., $\bm{A}\bm{x} + \bm{b}$.
An affine map $ \bm{z} = \bm{A}\bm{x} + \bm{b}$ maps a Gaussian $\bm{x} \sim \mathcal{N}(\bm{\mu}, \bm{\Sigma})$ onto a Gaussian $\bm{z} \sim \mathcal{N}(\bm{A}\bm{\mu} + \bm{b}, \bm{A}^T\bm{\Sigma}\bm{A})$.

Here we find an affine map $\bm{M}$ based on the linear model $\bm{A}_L$ that provides an approximation of the non-linear model $\bm{A}(\bm{x})$ for parameters $\bm{x}$ near the posterior mean $\bm{\mu}_{\bm{x}|\bm{y}}$.
\begin{figure}[htb!]
	\centering
	\begin{tikzpicture}
		\node[rectnode] at (0,0) (Oy)    {$\bm{y}$};
		\node[roundnode2] at (0,-2) (x)     {$\bm{x}$};
		\node[roundnode2] at (1.75,-4) (V)    {$\bm{V}$};%{$\bm{A}_{NL}\bm{x}$};
		\node[roundnode2] at (-1.75,-4) (W)    {$\bm{W}$};%{$\bm{A}_L\bm{x}$};
		\draw[->, very thick] (Oy.south) -- (x.north); 
		\draw[->, very thick] (x) -- (V); 
		\draw[->, very thick] (x) -- (W); 
		\draw[->, very thick] (W.east) --  (V.west); 
		\node[align=center] at (1,0) (l1) {data};
		\node[align=center] at (3.5,-2) (f2) {ozone profiles from $\pi(\bm{x}|\bm{y}) $};
		\node[align=center] at (1.75,-1) (l1) {$\pi(\lambda , \gamma  | \bm{y})$ with $\bm{A}_L$};
		
		
		\node[align=center] at (-4.25,-4) (f3) {linear forward model};
		\node[align=center] at (4.75,-4) (f4) {non-linear forward model};
		\node[align=center] at (0,-5) (f5) {$\bm{A}(\bm{x}) \approx \bm{M A}_L \bm{x}$ };
		
		\node[align=center] at (0,-4) (f5) {affine map \\ $\bm{M}$};
		
	\end{tikzpicture}
	\caption[Strategy to find affine map.]{The strategy to find the affine map consists of evaluating the marginal posterior for ozone $\pi(\lambda , \gamma  | \bm{y})$ for the linear forward model. For some ozone samples from the posterior the noise-free linear data $\bm{A}_L \bm{x}$ is calculated to form $\bm{W}$ and similarly $\bm{V}$ is composed using the non-linear model $\bm{A}(\bm{x})$. An affine map $\bm{M}$ approximates the non-linear model from the linear model.}
	\label{fig:affinStrat}
\end{figure}

\section{Finding an Affine Map}
We find an affine map by creating the vector spaces $\bm{W}$ based on the linear forward model and $\bm{V}$ based on the non-linear forward model with ground truth pressure and temperature.
More specifically $m-1$ samples $\bm{x}^{(j)} \sim \pi(\bm{x}|\bm{y})$, for $j = 2, \dots,m$, from the posterior and the posterior mean $\bm{\mu}_{\bm{x}|\bm{y}}$ generate,
\begin{align*}
	\bm{W} = \begin{bmatrix}
		\vert& \vert&   &  \vert & & \vert \\
		\bm{A}_{L}  \bm{\mu}_{\bm{x}|\bm{y}} & \bm{A}_{L}  \bm{x}^{(2)}   &  \cdots& \bm{A}_{L} \bm{x}^{(j)} &  \cdots & \bm{A}_{L} \bm{x}^{(m)} \\
		\vert& \vert&   &  \vert & & \vert 
	\end{bmatrix}
	\in \mathbb{R}^{m \times m}
\end{align*}\noindent and
\begin{align*}
	\bm{V} = \begin{bmatrix}
		\vert& \vert&   &  \vert & & \vert \\
		\bm{A}(\bm{\mu}_{\bm{x}|\bm{y}} ) & \bm{A}(\bm{x}^{(2)}) &  \cdots& \bm{A}(\bm{x}^{(j)}) &  \cdots & \bm{A} (\bm{x}^{(m)})  \\
		\vert&\vert&   &  \vert & & \vert 
	\end{bmatrix} = 
	\begin{bmatrix}
		\begin{array}{ccc}
			\horzbar & v_{1} & \horzbar \\
			& \vdots    &          \\
			\horzbar & v_{j} & \horzbar \\
			& \vdots    &          \\
			\horzbar &v_{m} & \horzbar
		\end{array}
	\end{bmatrix}\in \mathbb{R}^{m \times m} \, .
\end{align*}
Then the non-linear forward model is approximated as 
\begin{align}
	\bm{A}(\bm{x}) \approx \bm{M A}_L \bm{x} \, , \label{eq:AffineM}
\end{align}
where we solve $v_j =r_j \bm{W}$ for each row $r_j$ in
\begin{align*}
	\bm{V}\bm{W}^{-1} = \bm{M} =
	\begin{bmatrix}
		\begin{array}{ccc}
			\horzbar & r_{1} & \horzbar \\
			& \vdots    &          \\
			\horzbar & r_{j} & \horzbar \\
			& \vdots    &          \\
			\horzbar &r_{m} & \horzbar
		\end{array}
	\end{bmatrix}\, \in \mathbb{R}^{m \times m} .
\end{align*}
using the Python function \texttt{numpy.linalg.solve}.
This is feasible since every noise-free measurement is independent of each other, and then every row $v_j$ of $\bm{V} \in \mathbb{R}^{m \times m}$ is independent of each other as well.
For an $\bm{x} = \bm{\mu}_{\bm{x}|\bm{y}} + \Delta \bm{x}$ we rewrite Eq.~\ref{eq:AffineM} to
\begin{align}
	\bm{A}(\bm{x})  \approx \underbrace{  \bm{M A}_L  \bm{\mu}_{\bm{x}|\bm{y}} }_{= \bm{A}( \bm{\mu}_{\bm{x}|\bm{y}} )  }+  \underbrace{\bm{M A}_L  \Delta \bm{x} }_{= \bm{A}^{\prime}( \bm{\mu}_{\bm{x}|\bm{y}} )  \Delta \bm{x} }\, \\
	=    \underbrace{ \bm{A}^{\prime}( \bm{\mu}_{\bm{x}|\bm{y}} ) \bm{x}}_{ \bm{A}\bm{x}}  +  \underbrace{ \bm{A}( \bm{\mu}_{\bm{x}|\bm{y}} )  - \bm{A}^{\prime}( \bm{\mu}_{\bm{x}|\bm{y}} ) \bm{\mu}_{\bm{x}|\bm{y}}}_{  \bm{b}}
\end{align}
to show that $ \bm{M}:\bm{A}_L\bm{x} \rightarrow \bm{A}(\bm{x})$ is an affine map.
%	\centering
%	\begin{tikzpicture}
%		\node[rectnode] at (2,-4) (NL)    {$\bm{V}$};
%		\node[rectnode] at (-2,-4) (L)    {$\bm{W}$};
%		\draw[<-, very thick] (NL.west) -- (L.east); 
%		\node[align=center] at (-5.5,-4) (f3) {linear forward model};
%		\node[align=center] at (5.5,-4) (f4) {non-linear forward model};
%		\node[align=center] at (0,-5) (f5) {$\bm{A}(\bm{x})  \approx \bm{M A}_L  \bm{x} $ };
%		\node[align=center] at (0,-4) (f5) {affine Map \\ $\bm{M}$};
%	\end{tikzpicture}
%	\caption[Schematics of the affine map]{This figure shows the schematic representation of how the affine map $\bm{M}$ approximates the non-linear forward model. Here, $\bm{V}$ contains values produced by the linear forward model, and $\bm{W}$ contains the corresponding values from the non-linear forward model. Both $\bm{V}$ and $\bm{W}$ are affine subspaces over the same field. The affine map $\bm{M}$ projects elements from the linear forward model space $\bm{V}$ onto their counterparts in the non-linear forward model space $\bm{W}$. More specifically, the non-linear noise-free data vector $\bm{A} (\bm{x}) $ is approximated by the affine map and the linear forward model so that  $\bm{A} (\bm{x})  \approx  \bm{M A}_L  \bm{x}$.}
%	\label{fig:AffSchem}
%\end{figure}


%Alternatively, one can also determine this map using other methods, e.g. machine learning methods or matrix inversion, which in our case did not give better results.

The relative RMS difference $\lVert \text{vec}(\bm{M}\bm{W}) - \text{vec}(\bm{V})  \rVert_{L^2} / \lVert \text{vec}(\bm{M}\bm{W}) \rVert_{L^2} $ between the mapped linear noise-free data and the non-linear noise-free data is approximately $0.001\%$.
This is much smaller than the relative RMS difference between $\bm{W}$ and $\bm{V}$ of about $1\%$.
Here $\text{vec}(\bm{V})$ vectorises the matrix $\bm{V}$.
Fig.~\ref{fig:MapAsses} shows the mapping for one posterior ozone sample with a relative RMS error~$\approx0.01\%$.
This posterior ozone sample has not been used to create this mapping; in other words, this is an unseen event not occurring in the training data.
Consequently, from here onwards, we use the approximated forward map.
\begin{figure}[ht!]
	\centering
	%\includegraphics{SampMapAssesment.png}
	\includegraphics{SampMapAssesmentTT.png}
	\caption[Assessment of affine map.]{Assessment of how well the affine map $\bm{M}$ approximates noise-free non-linear data $\bm{A}(\bm{x})$ (red circles) from noise-free linear data $\bm{A}_L\bm{x}$ (grey stars). The approximated noise-free data (black stars) has a relative RMS error of $\approx 0.01\%$ compared to the true non-linear noise-free data.
		The ozone profile $\bm{x}$ to generate this noise-free data has not been used to create the affine map.}
	\label{fig:MapAsses}
\end{figure}
\clearpage

\section{Marginal and Full Conditional Posterior -- Ozone}
The exact same setup and procedure as in Sec.~\ref{sec:FirstO3Post} is used to evaluate the marginal posterior and then the full conditional posterior of ozone, but with the approximated forward model.

The MWG is initialised at the mode of the marginal posterior $\pi(\lambda,\gamma| \bm{y})$ as in Eq.~\ref{eq:MargPostAppl}.
The functions $f(\lambda)$ and $g(\lambda)$ are approximated around the mode as in Sec.~\ref{subsec:FirstMargPost} (see Eq.~\ref{eq:fAprox} and Eq.~\ref{eq:gAprox}).
We take $N = 10000$ plus $N_{\text{burn-in}} = 100$ steps in $\approx 0.5$s.
The IACTs $\tau_{\text{int}, \gamma} \approx 5.2 \pm 0.3$ and $\tau_{\text{int}, \lambda} = 11 \pm 1 $ (see Fig.~\ref{fig:IATCSecO3gam} and Fig.~\ref{fig:IATCSecO3lam}) are twice the values provided by~\cite{drikHesse} and similar to the previously calculated values (see Sec.~\ref{subsec:FirstMargPost}).

The TT approximation of the marginal posterior is obtained using 400 function evaluations (same grid; same number of ranks; see Sec.~\ref{subsec:FirstMargPost}) within $\approx 0.02$s.
The relative RMS error of the TT approximation over the whole grid is similar to the relative RMS error of $\pi(\lambda,\gamma| \bm{y})$ due to the approximations of $f(\lambda)$ and $g(\lambda)$.
Both errors are $\approx 7\%$.

The MWG samples as well as the marginal functions from the TT approximation (calculated as in Sec.~\ref{subsec:FirstMargPost}) are plotted in Fig.~\ref{fig:MargPostHistTT}.
\begin{figure}[ht!]
	\centering
	\includegraphics{secMargO3Res.png}
	\caption[Marginal posterior histograms and TT approximation as well as hyper-prior distribution.]{The TT approximation of the marginal posterior plotted in orange and the samples of the MWG algorithm displayed as histograms. The prior distribution is plotted as dotted line. For each $\lambda$ and $\gamma$ sample from the marginal posterior $\delta = \lambda \gamma$ is calculated.}
	\label{fig:MargPostHistTT}
\end{figure}
\clearpage

Again, the full posterior mean $\bm{\mu}_{\bm{x}|\bm{y}}$ (see Eq.~\ref{eq:MeanInt}) and covariance matrix $\bm{\Sigma}_{ \bm{x}|\bm{y}}$ (see Eq.~\ref{eq:CovInt}) are calculated as weighted expectation over a $20\times 20$ grid.
The resulting posterior mean and STD, and one sample of $\pi(\bm{x}|\bm{y})$, which represents a feasible solution to this inverse problem, are plotted in Fig.~\ref{fig:O3SolplsReg}.
The ground truth lies within the STD around the posterior mean, except for the peak at around $80$km.
Compared to the previously calculated posterior mean and covariance based on the linear model $\bm{A}_L$ (see Fig.~\ref{fig:O3Samp}), the posterior distribution based on the approximated $\bm{M A}_L$ does not differ significantly.
This is expected since the difference between the linear and non-linear forward map of $\approx 1 \%$ is small.
\begin{figure}[ht!]
	\centering
	\includegraphics{SecRecResinclRegandSampl.png}
	%\includegraphics{SecRecResinclReg.png}
	\caption[Full posterior mean and variance of ozone and the regularised solution compared to the ground truth.]{Full posterior mean and variance and one ozone sample from the full posterior. These results are based on the approximated forward model $\bm{M}\bm{A}_L$.}
	\label{fig:O3SolplsReg}
\end{figure} 


%Then we are able to define a linear model $ \bm{A} \coloneqq \bm{M} \bm{A}_{L}$, which approximates the non-linear model.
%Here, we give a brief introduction to affine maps.
%
%An affine map is any linear map between two vector spaces or affine spaces, where an affine space does not need to preserve a zero origin (see~\cite[Def. 2.3.1]{berger2009geometry}). \textcolor{red}{no it's not}
%In other words, an affine map does not need to map to the origin of the associated vector space.
%An affine map is a linear map on vector spaces, including a translation, or, in the words of my supervisor, C. F., a Taylor series of first order. \textcolor{red}{it is not a linear map , the origin of the domain to the origin of the range, fix this sentence - -it makes no sense.}
%For more information on affine spaces and maps, we refer to the books~\cite{berger2009geometry, katsumi1994affine}. \textcolor{red}{don't say that, say that it is a linear map plus a constant.}
%
%\textcolor{red}{these are horrible books. No wonder you have a garbled idea and description. All affine maps look like y = Ax+b where A is linear and b is a constant. What's so difficult about that?
%	I asked a LLM:
%	Please give a simple definition of an affine map (between two vector spaces)
%	Sure! A simple definition of an affine map between two vector spaces is:
%	An affine map is a function between vector spaces that preserves straight lines and has the form
%	$>f(x)=Ax+b>$
%	where A is a linear transformation and b is a fixed vector.
%	So it’s like a linear map, but with a shift. }