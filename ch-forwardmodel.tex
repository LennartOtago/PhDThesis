\chapter{Forward Model}
\label{ch:formodel}
In this chapter we present the forward model we apply all our methodology on. We follow the MIPAS handbook \cite{mipas2000handbook} and simulate data according to a atmosphere in local thermodynamic and and assume a measurement instrument with infinite spectral resolution and no pointing errors.


The forward model is based on a satellite measuring thermal radiation of gas molecules along its line of sight by pointing through the atmosphere to the edge(limb) of the atmosphere, known as limb sounding, as shown in Figure~\ref{fig:LIMB}.
One measurement $y_j$, of a stationary satellite is given by the path integral through the atmosphere along the line of sight.
For each measurement $j=1,2,\ldots,m$ of a data set, we can define a tangent height $h_{\ell_j}$ as the shortest distance along the line of sight to the earth.
\begin{figure}[ht!]
	\centering
	\input{LIMB.pdf_tex}
	\caption[Schematic of measurement and analysis geometry.]{Schematic of measurement and analysis geometry, not to scale.
		The stationary satellite, at a constant height $h_\text{sat}$ above Earth, takes $m = 41$ measurements along its line-of-sight defining by the line $\Gamma_j$.
		Each measurement has a limb height $\ell_j$, $j=1,2,\dots,m$ defined as the closest distance of $\Gamma_j$ to the Earth surface.
		Between $h_{L,0} = 7$km and $h_{L,n} = 83.3$km, the stratosphere is discretised into $n =44$ layers as illustrated by the solid green lines.}
		\label{fig:LIMB}
\end{figure}

Targeting the thermal radiation at one wave number $\nu$ of one specific molecule, the $j^\text{th}$ measurement, is modelled by the radiative transfer equation (RTE)~\cite{mipas2000handbook}
\begin{align}
	\label{eq:RTE} 
	y_j =   \int_{\Gamma_j}  B(\nu,T) k(\nu, T)   \frac{p(T)}{k_{\text{B}} T(r)}  x(r)  \tau(r) \text{d}r + \eta_j \, \\
	\tau(r) = \exp{ \Bigl\{ - \int^{r}_{r_\text{obs}}  k(\nu, T)   \frac{p(T)}{k_B T(r^{\prime})}  x(r^{\prime}) \text{d}r^{\prime} \Bigr\} } \, ,\label{eq:absRTE} 
\end{align}
where the path from the satellite along the line-of-sight of the $j^\text{th}$ pointing direction is $\Gamma_j$ and the ozone concentration at distance $r$ from the radiometer is $x(r)$ plus some noise $\eta_j$.
Within the atmosphere the number density $p(T) / (k_{\text{B}} T(r))$ of molecules is dependent on the pressure $p(T)$, the temperature $T(r)$, and the Boltzmann constant $k_{\text{B}}$.
The factor $\tau(r)\leq 1$ accounts for re-absorption of the radiation along the line-of-sight, which makes the RTE non linear.
The absorption constant
\begin{align}
	k(\nu, T) = L(\nu, T_{\text{ref}}) \frac{Q(T_{\text{ref}})}{Q(T)} \frac{ \exp{\{ - c_2 E^{\prime \prime} / T\}} }{\exp{\{ - c_2 E^{\prime \prime} / T_{\text{ref}} \}}} \frac{ 1- \exp{\{ - c_2 \nu  / T \}} }{1 - \exp{\{ - c_2 \nu / T_{\text{ref}} \}}}
\end{align}
is depend on the line intensity $L(\nu, T_{\text{ref}})$ at reference temperature $T_{\text{ref}} =296K $, the lower-state energy of the transition $ E^{\prime \prime} $, the second radiation constant $c2=1.4387769\text{cmK}$ all provided by the HITRAN database \cite{gordon2022hitran2020}.
The total internal partition function for the lower-state energy is
\begin{align}
	Q(T )= g^{\prime \prime} \exp{\{ - \frac{ c_2 E^{\prime \prime} }{T}\}} \, ,
\end{align}
with the statistical weight $ g^{\prime \prime}$ (also called the degeneracy factor) accounting for the molecules non-rotational and rotational energy states, see \cite{vsimevckova2006einstein}.
Under the assumption of local thermodynamic equilibrium (LTE) the black body radiation act as a source function
\begin{align}
	B(\nu,T)   = \frac{2 h c^2 \nu^3}{\exp{\{\frac{hc\nu}{k_B T}\}}-1}\, ,
\end{align}
with Planck's constant $h$ and velocity of light $c$ \cite{}.
For fundamentals on the Radiative transfer equation we recommend \cite[Chapter 1]{rybicki2000rte}.

%The absorption constant $k(\nu, T)$ for a single gas molecule at a specific wave number $\nu$ is calculated according to the HITRAN database \cite{gordon2022hitran2020} and acts as a source function when multiplied with the black body radiation $B(\nu,T)$, given by Planck`s law~\cite{rybicki2000rte}.
%We calculate the source function $B(\nu, T)$ and the absorption constant $ k(\nu, T)$ as follows.
%assumption local thermaodynamic equilibrium
%For one species at one specific wave-number the weighted absorption constant becomes
%\begin{align}
%	\overline{k(\nu, T, r)}    = \sum_{m=1}^{molec} k_m(\nu, T) x_m(r) =  k(\nu, T) x(r) \, ,
%\end{align}
%with the volume mixing ratio of ozone $x(r)$ at location $r$. 


To enable matrix-vector multiplication, we discretise the atmosphere in $n$ layers, where the $i^\text{th}$ layer is defined by two spheres of radii $h_{L,i-1} < h_{L,i}$, for $i = 1, \dots, n$, with $h_0$ and $h_{L,n} $.
Then we can discretise the ozone, pressure and temperature profiles as a function of height, where in between the heights $h_{L,i-1}$ and $h_{L,i}$, each of the ozone concentration $x_{i}$, the pressure $p_{i}$, the temperature $T_{i}$, as well as the thermal radiation is assumed to be constant.
Above $h_{L, n} = $ and below $h_{L,0} = $, the ozone concentration is set to zero, so no signal can be obtained.
Depending on the parameter of interest, which is either the ozone volume mixing ratio $\bm{x} =\{x_1,x_2,\ldots,x_n\} \in \mathbb{R}^{n}$ or the fraction of pressure and temperature $\bm{p/T}= \{p_1/T_1,p_2/T_2,\ldots,p_n/T_n\} \in \mathbb{R}^{n} $, we solve the integral in Eq.~\eqref{eq:RTE} using the trapezoidal rule so that we can rewrite the integral to a vector multiplication $\bm{A_{j}}(\bm{x},  \bm{p},\bm{T}) \, \bm{x} $ or $\bm{A_{j}}(\bm{x},  \bm{p},\bm{T}) \, \bm{p}/ \bm{T} $, where the non-linear absorption $\tau(r)$ is included in $\bm{A_{j}}(\bm{x},  \bm{p},\bm{T})$.
Here, the row vector $\bm{A_{j}}(\bm{x},  \bm{p},\bm{T}) \in \mathbb{R}^{n}$  defines a Kernel for each measurement so that the data vector
\begin{align}
	\bm{y} = \bm{A}(\bm{x},  \bm{p},\bm{T}) \, \bm{x} + \bm{\eta}= \bm{A}(\bm{x},  \bm{p},\bm{T}) \,
	\frac{ \bm{p}}{\bm{T}} + \bm{\eta} \, .
\end{align}
can be written as a matrix-vector multiplication, with the matrix $\bm{A}(\bm{x},  \bm{p},\bm{T}) \in \mathbb{R}^{m \times n}$ and the noise vector $\bm{\eta} \in \mathbb{R}^{m}$.
Note, for simplicity, we do not explicitly specify whether $\bm{A}(\bm{x}, \bm{p}, \bm{T})$ is constructed conditioned on $\bm{x}$ or $\bm{p}/\bm{T}$ and we ignore temperature dependencies of the pressure, absorption constant or the black body radiation.

Since the measurement process includes absorption $\tau(r)$, which reduces measurements only slightly, we classify the inverse problem as weakly non-linear. 
Hence, we can approximate the non-linear forward model $\bm{A}(\bm{x},  \bm{p},\bm{T})$ with a map $\bm{M}$ and the linear forward model $\bm{A}_L$, so that $\bm{A}(\bm{x},  \bm{p},\bm{T}) \approx \bm{M} \bm{A}_L $.
Here, $\bm{A}_{L,j} $ of matrix $\bm{A}_L \in \mathbb{R}^{m \times n}$ is defined by the linear forward model, where absorption is neglected, e.g. set $\tau = 1$ in Eq.~\eqref{eq:absRTE}. 
Then each entry in the row vector $\bm{A}_{L,j} $ is either defined by $ B(\nu) k(\nu)   \frac{\bm{p}}{k_{\text{B}} \bm{T}}  \text{d}r$ or $B(\nu) k(\nu)   \frac{\bm{x}}{k_{\text{B}}}  \text{d}r$, as in Eq.~\eqref{eq:RTE}.
This poses a linear inverse problem with the forward map defined by the matrix $\bm{A} = \bm{M} \bm{A}_L$, where $\bm{M}$ is, more specifically, an affine map.


\textcolor{red}{$h_{L,0}$ does not influences the values for $p_{L,O}/T_{L,0}$}
\textcolor{red}{dont include bend of the line integral}

%\begin{figure}[ht!]
%	\centering
%	\scalebox{0.9}{\input{FirstLIMB.pdf_tex}}
%	\caption[General schematics of measurement setup]{This figure illustrates a limb-sounding measurement setup, specifically how the line of sight of a satellite at altitude $h_{\text{obs}}$ is partitioned according to a discretised atmospheric model. The atmosphere is divided into $n$ layers, allowing the line of sight $\Gamma_j$ to be discretised into segments $\Delta r_i$ for $i = \ell_j, \dots, n$.
%		Here, $\ell_j \in \mathbb{N}$ denotes the index corresponding to the tangent height $h_{\ell_j}$ relative to the Earth's radius $R_E$. This setup forms the basis for the numerical solution of the integral in Eq.~\ref{eq:RTE}, known as the radiative transfer equation.}
%	\label{fig:FirstLIMB}
%\end{figure}