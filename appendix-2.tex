\chapter{Correlation Structure}
\label{ap:Correlatation}
The book by Rue and Held shows the correlation structur between the hyperparameter $\mu$ and the laten field $\bm{x}$
We consider the hierarchal formuation 
\begin{align}
	\mu \sim \mathcal{N}(0,1)\\
	\bm{x}|\mu \sim \mathcal{N}(\mu\bm{1},\bm{Q}^{-1})
\end{align}
Gibbs sampler Since both full conditional
\begin{align}
	\mu^{(k)} | \bm{x}^{(k)} &\sim \mathcal{N} \Bigg(\frac{\bm{1}^T\bm{Q}\bm{x}^{(k-1)}}{1 + \bm{1}^T\bm{Q}\bm{1}}, (1 + \bm{1}^T\bm{Q}\bm{1})^{-1} \Bigg)\\
	\bm{x}^{(k)} | \mu^{(k)} &\sim \mathcal{N} (	\mu^{(k)}\bm{1}, \bm{Q}^{-1})
\end{align}

\begin{figure}[ht!]
	\centering
	
	\caption[Correlation structur]{Figure 4.1 Figure (a) shows the marginal chain for µ over 1000 iterations of
		the marginal chain for the hyperparmeter with a specuifuc autoregressive proces defined in $\bm{Q}$ The algorithm
		updates successively $\mu$ and $x$ from their full conditionals. Figure (b) displays
		the pairs ($\mu(k)$ ,  $1^T\bm{Q}x(k)$ ), with$\mu(k)$ on the horizontal axis. The slow mixing
		(and convergence) of $\mu$is due to the strong dependence with $1^T\bm{Q}x(k)$  as only
		horizontal and vertical moves are allowed. The arrows illustrate how a joint
		update can improve the mixing (and convergence)}
	\label{fig:FirstDAG}
\end{figure}

Our solution is to update $(\mu, x)$ jointly, where effectefally the 

Note that only the marginal density of µ is needed in (4.6): Since we
sample x from its full conditional, we effectively integrate x out of the
joint density $\pi(\mu, x)$.


\chapter{Mesure theroy}
\label{ap:Correlatation}
Recall that Asumme that the triple $(\Omega, \mathcal{F} , \mathbb{P})$ is called a probability space over the whole sample space $\Omega$ with the collection $\mathcal{F}$ of very countable subset $\{ A _n \}_{n\in \mathbb{N}}$, where a Event $A_n$ is a subset of $\Omega$, $A_n \subseteq  \Omega$, and a map $\mathbb{P} : \mathcal{F} \longrightarrow \mathbb{R}$. 
We assume $\mathbb{P}$ fullfils probality measure and  $\mathcal{F}$ is a $\sigma $-algebra see Appendix \ref{}.
We call $\mathbb{P}(A)$ the probaility of an event $A \subseteq \mathcal{F}$ 
\section{sigma alrgbea}
A collections of subsets $\mathcal{F}$ is called sigma algreab if
\begin{itemize}
	\item $\emptyset, \Omega \in \mathcal{F} $
	\item if $A \in \mathcal{F} $ then $A^C \coloneqq A / \Omega \in A$
	\item if $A_1 , A_2, \dots  \in \mathcal{F} $ then $ \bigcup_{j=1}^{\infty} A_j \in A$
\end{itemize}
\section{probailty measure}

a probailty measure has to full fil the folowing criterion 
\begin{itemize}
	\item $\mathbb{P}(\Omega) = 1$ and $\mathbb{P}(\emptyset) = 1$
	\item $\mathbb{P}(A) \in [0,1]$
	\item $\mathbb{P}(A \cup B) =\mathbb{P}(A) + \mathbb{P}(B) $ if $A, B$ are disjoint or $A\cap B = \emptyset$
	\item $\mathbb{P}(\bigcup_{j = 1}^{\infty} A_j) = \sum_{j=1}^{\infty} \mathbb{P}(A_j)$ if we have pairwise dijoint sets or $A_i \cap A_j = \emptyset$ for $i \neq j $
\end{itemize}