\the\columnwidth
\chapter{Theoretical and Technical Background}
\label{ch:background}
In this chapter, we provide brief introductions and derivations to the methods used in this thesis, as well as references for more details. We keep it as general as possible, as the expressions specifically tailored towards the forward map will be presented in the results Chapter~\ref{ch:res}, but without derivations.
We begin by introducing a general hierarchical Bayesian approach to a linear inverse problem.
Next, we provide a small introduction to sampling methods, more specifically, the essentials of Markov-Chain Monte Carlo (MCMC) methods.
Lastly, we explain how we approximate functions using a Tensor-Train (TT) approach, which enables us to calculate marginals from the posterior distribution cheaply.
Lastly, we provide some background information on affine maps and the Tikhonov regularisation method.

\section{Bayesian Inference}
\label{sec:bayes}
\begin{figure}[ht!]
	\centering
	\begin{tikzpicture}
		\node[roundnode2] at (0,3.5) (th)    {$\bm{\theta}$};
		\node[roundnode2] at (0,1.5) (x)    {$\bm{x}$};
		\node[roundnode2] at (0,-1.5) (u)    {$\Omega$};
		\node[rectnode] at (0,-3.5) (y)    {$\bm{y}$};
		
		\draw[->, very thick] (th.south) -- (x.north); 
		\draw[->, very thick, mydotted] (x.south) -- (u.north); 
		\draw[->, very thick] (u.south) -- (y.north); 
		\draw[->, very thick] (th) edge[bend right=60] (y);  
		
		\node[align=center] at (2.8,3.5) (tht) {$\sim \pi_{\bm{\theta}}(\cdot) $ hyper-parameters};
		\node[align=center] at (2.4,1.5) (xt) {$\sim \pi_{\bm{x}}(\cdot|\bm{\theta}) $ parameters};
		\node[align=center] at (2.5,0) (At) {$\{\bm{Ax}\}$ "noise free data"};
		\node[align=center] at (3,-1.5) (ut) {space of all measurables};
		\node[align=center] at (2,-3.5) (yt) {$\sim \pi_{\bm{y}}(\cdot|\bm{\theta},\bm{x})$ data};
		\node[align=center] at (-3,0) (nt) {noise};
	\end{tikzpicture}
	\caption[Bayesian Inference DAG]{The directed acyclic graph (DAG) for a linear inverse problem visualises statistical dependencies as solid line arrows and deterministic dependencies as dotted arrows.
	The hyper-parameters $\bm{\theta}$ are distributed as the hyper-prior distribution $\pi(\bm{\theta})$.
	The prior distribution $ \pi_{\bm{x}}(\cdot|\bm{\theta})$ for the parameter $\bm{x}$ and the noise are statsitically dependent on have some on those hyper-parameters.
	 Then a parameter $\bm{x} \sim \pi_{\bm{x}}(\cdot|\bm{\theta})$ is mapped onto the space of all measurables $\bm{u}=\bm{Ax}$ deterministically through the linear forward model $\bm{A}$.
	From the space of all measurable noise free data we observe a data set $\bm{y} = \bm{Ax} + \bm{\eta}$ with some random noise $ \bm{\eta} \sim \pi_{\bm{\eta}}(\cdot|\bm{\theta})$, which determines the likelihood function $\pi(\bm{y}|\bm{\theta}, \bm{x})$. }
	\label{fig:FirstDAG}
\end{figure}

Assume we observe some data
\begin{align}
	\bm{y} = \bm{A} \bm{x} + \bm{\eta},
	\label{eq:LinDat}
\end{align}
based on a linear forward model $\bm{A}$, a unknown parameter $\bm{x}$ and some additive random noise $\bm{\eta}$.
Naturally, due to the noise, we have some uncertainty which we include in the modelling process through the likelihood function $\pi(\bm{y}|\bm{\theta},\bm{x})$ as well as other relevant information about the measurement process.
We read $\pi(\bm{y}|\bm{\theta},\bm{x})$ as the distribution over $\bm{y}$ conditioned on $\bm{x}$ and the hyper-parameter $\bm{\theta}$.
Here $\bm{\theta}$ may account for multiple variables and is e.g. describing the distribution of the noise vector $\bm{\eta} \sim \pi_{\bm{\eta}}(\cdot|\bm{\theta})$ as well as the prior distribution $\pi(\bm{x}|\bm{\theta})$, which accounts for physical properties or functional dependences of $\bm{x}$.
Consequently we define a hyper-prior distribution $\pi(\bm{\theta})$, where $\pi(\bm{x}, \bm{\theta}) = \pi(\bm{x}|\bm{\theta}) \pi(\bm{\theta}) $.
Choosing these prior distributions is a delicate topic, as it shall not affect the posterior distribution 
\begin{align}
	\pi(\bm{x},\bm{\theta}|\bm{y}) = \frac{ \pi(\bm{y} | \bm{x}, \bm{\theta} ) \pi(\bm{x}, \bm{\theta})}{\pi(\bm{y})} \propto \pi(\bm{y} | \bm{x}, \bm{\theta} ) \pi(\bm{x}, \bm{\theta}) \, ,
\end{align}
which according to Bayes theorem gives us a distribution of $\bm{x}$ and $\bm{\theta}$ given (conditioned) on some data.
We can visualise this hierarchically ordered correlation structure between parameters as well as how distributions progress through a measurement process, using a directed acyclic graph (DAG), see Figure~\ref{fig:FirstDAG}.

The expectation of any function $h(\bm{x}_{\bm{\theta}})$, where $\bm{x}$ may depend on $\bm{\theta}$, is described as 
\begin{align}
	\text{E}_{\bm{x},\bm{\theta}|\bm{y}} [h(\bm{x}_{\bm{\theta}})] =  \underbrace{\int \int   h(\bm{x}_{\bm{\theta}}) \,  \pi(\bm{x}, \bm{\theta} | \bm{y} ) \, \text{d} \bm{x}  \, \text{d} \bm{\theta}}_{\bm{\mu}_{\text{int}}}   \label{eq:expPos} \, ,
\end{align}
where $\pi(\bm{x}, \bm{\theta} | \bm{y} )$ is the posterior distribution.
This may be a high-dimensional integral and computationally not feasible to solve.
Therefore the unbiased sample-based Monte Carlo estimate \cite{roberts2004general}
\begin{align}
	\label{eq:sampMean}
	\text{E}_{\bm{x},\bm{\theta}|\bm{y}} [h(\bm{x}_{\bm{\theta}})] \approx \underbrace{ \frac{1}{N} \sum_{k=1}^{N} h(\bm{x}^{(k)}_{\bm{\theta}})  }_{\bm{\mu}_{\text{samp}}} \, ,
\end{align}
for large enough $N$ (law of large numbers \cite[Chapter 17]{tweedie2009measprob}) is often used.
Here, the samples $\{\bm{x}^{(k)},\bm{\theta}^{(k)} \}\sim \pi_{\bm{x}, \bm{\theta}}(\cdot|\bm{y})$, for $k = 1, \dots, N$, form a sample set $\mathcal{M} =\{ (\bm{x},\bm{\theta})^{(1)}, \dots ,  (\bm{x},\bm{\theta})^{(N)} \}$.
Generating a representative sample set quickly from the posterior distribution often presents a significant challenge. This is mainly due to the strong correlations that usually exist between the parameters and hyper-parameters, as discussed by Rue and Held in \cite{rue2005gaussian} and illustrated in Appendix~\ref{ap:Correlation}.
If $\bm{x}$ can not be parametrised directly in terms of the hyper-parameters $\bm{\theta}$, i.e., $\bm{x}(\bm{\theta})$, it is beneficial to factorise the posterior distribution as
\begin{align}
	\pi(\bm{x}, \bm{\theta} |  \bm{y}) = \pi(\bm{x} |  \bm{\theta}, \bm{y}) \, \pi(\bm{\theta} |   \bm{y}), \label{eq:MTC}
\end{align}
into the conditional posterior $\pi(\bm{x} |  \bm{\theta}, \bm{y})$ over the latent field $\bm{x}$ and the marginal posterior 
\begin{align}
	\pi(\bm{\theta} |   \bm{y}) =  \frac{ \pi(   \bm{y} | \bm{\theta} ,\bm{x})  \pi( \bm{x} | \bm{\theta} )  \pi(\bm{\theta}) }{ \pi(\bm{x} | \bm{\theta} ,   \bm{y})   \pi( \bm{y})} \propto \frac{ \pi(   \bm{y} | \bm{\theta} ,\bm{x})  \pi( \bm{x} | \bm{\theta} )  \pi(\bm{\theta}) }{ \pi(\bm{x} | \bm{\theta} ,   \bm{y}) }
\end{align}
over the hyper-parameters $\bm{\theta}$.
This approach, known as the marginal and then conditional (MTC) method, is particularly advantageous when $\bm{x}$ is high-dimensional (e.g., $\bm{x} \in \mathbb{R}^n$ with $n \geq 45$), while $\bm{\theta}$ is low-dimensional (e.g., two-dimensional).
Applying the law of total expectation~\cite{champ2022generalizedlawtotalcovariance}, Eq.~\eqref{eq:expPos} becomes
\begin{align}
	\mathbb{E}_{\bm{x} |  \bm{y}} [h(\bm{x})] 
	= \mathbb{E}_{\bm{\theta} |  \bm{y}} \left[ \mathbb{E}_{\bm{x} |  \bm{\theta}, \bm{y}} [h(\bm{x}_{\bm{\theta}})] \right] 
	= \int \mathbb{E}_{\bm{x} |  \bm{\theta}, \bm{y}} \left[ h(\bm{x}_{\bm{\theta}}) \right] \, \pi(\bm{\theta} |  \bm{y}) \, \mathrm{d}\bm{\theta},
	\label{eq:fullCond}
\end{align}
where, in the case of a linear-Gaussian Bayesian hierarchical model, both the marginal distribution and the inner expectation $\mathbb{E}_{\bm{x} |  \bm{\theta}, \bm{y}} \left[ h(\bm{x}_{\bm{\theta}}) \right]$ are well defined see Result chapter.
Furthermore, the central limit theorem states that the samples mean $ \bm{\mu}^{(i)}_{\text{samp}} $, of independent samples sets $\mathcal{M}_i$ for $i = 1, \dots, n$ of any distribution, converge in distribution to a normal distribution so that
\begin{align}
	\sqrt{n} (\bm{\mu}^{(i)}_{\text{samp}} -  \bm{\mu}_{\text{int}} ) \overset{\mathcal{D}}{\longrightarrow} \mathcal{N} (0,\sigma^2) \text{\cite{geyer1992practical}},
\end{align}
and if $\sigma^2 < \infty$ the Monte-Carlo error $\bm{\mu}^{(i)}_{\text{samp}} -  \bm{\mu}_{\text{int}} $ is bounded.

\subsubsection{On the Monte-Carlo Error and Integrated Autocorrelation time}
To assess the error $\sigma^2$, we ignore systematic error due to initialisation bias (burn-in period), but we have to take into account that samples produced by any system or algorithm are correlated.
In general, the error of a Monte-Carlo-based estimate from a sample set $\mathcal{M}_i$ is:
\begin{align}
	(\sigma^{(i)})^2 = \text{var}(\bm{\mu}^{(i)}_{\text{samp}} ) =  \text{var}(\text{E}_{\bm{x},\bm{\theta}|\bm{y}} [h(\bm{x}_{\bm{\theta}})]) = \Bigg( \frac{1}{N} \sum_{k=1}^{N} h(\bm{x}^{(k)}_{\bm{\theta}}) - \bm{\mu}^{(i)} \Bigg)^2 \, .
\end{align}
Expanding this summation, we see that
\begin{align}
	(\sigma^{(i)})^2	= \frac{1}{N^2} \sum_{k,s=1}^{N} C(k-s)
\end{align}
with the auto correlation coefficient $C(k-s) =  \big( h(\bm{x}^{(k)}_{\bm{\theta}}) - \bm{\mu}^{(i)} \big) \big(h(\bm{x}^{(s)}_{\bm{\theta}}) - \bm{\mu}^{(i)} \big)$ and define the sample auto correlation function
\begin{align}
	\frac{C(0)}{N} \sum_{k,s=1}^{N} \frac{C(k-s)}{C(0)} \approx \text{var}(h(\bm{x}_{\bm{\theta}}) ) \sum_{t = - \infty }^{\infty} \rho(t)
\end{align}
with the normalised auto correlation coefficient $\rho(k-s) =  C(k-s)/ C(0)$ at lag $k-s$, where $C(0) = \text{var}(h(\bm{x}_{\bm{\theta}}) )$ for $k = s$.
Then the estimate for the Monte-Carlo error is:
\begin{align}
	(\sigma^{(i)})^2   \approx  \frac{\text{var}(h(\bm{x}_{\bm{\theta}}) )}{N} \underbrace{\sum_{t = - \infty }^{\infty} \rho(t)}_{	2\tau_{\text{int}} } = \text{var}(h(\bm{x}_{\bm{\theta}})) \frac{ 2 \tau_{\text{int}} }{N} \, , \label{eq:MCerr}
\end{align}
where we define the integrated autocorrelation time (IACT) $\tau_{\text{int}}$ as in \cite{Sokal1997} and \cite{}, which provides a good estimate on how many steps the sampling algorithm needs to take to produce one independent sample.
More specifically, the effective sample size $\frac{ 2 \tau_{\text{int}} }{N}$ gives an estimate of how efficient a sampler is.
We calculate the IACT using the Python implementation of \cite{WOLFF2004143}, accessed \cite{wolff2004monte}, which also provides errors on the IACT, which helps us to assess convergence of the chain.


\section{Sampling Methods}
\label{sec:sampling}
In this section we present the sampling methods used in this thesis and show how these methods draw samples $ \mathcal{M} = \{ (\bm{x}, \bm{\theta} )^{(1)}, \dots, (\bm{x}, \bm{\theta} )^{(k)}, \dots, (\bm{x}, \bm{\theta})^{(N)} \} \sim \pi(\bm{x}, \bm{\theta} |  \bm{y})$ from the desired target distribution, so that we can apply sample-based estimates as in Eq.~\ref{eq:sampMean}.
Here, $\mathcal{M}$ denotes a Markov chain, where each new sample $(\bm{x}, \bm{\theta})^{(k)}$ is only affected by the previous one, $(\bm{x}, \bm{\theta})^{(k-1)}$.
Markov chain Monte Carlo (MCMC) methods generate such a chain $\mathcal{M}$ using random (Monte Carlo) proposals $(\bm{x}, \bm{\theta})^{(k)} \sim q( \cdot |  (\bm{x}, \bm{\theta})^{(k-1)})$ according to a proposal distribution conditioned on the previous sample (Markov), where ergodicity of the chain $\mathcal{M}$ is a sufficient criterion for using sample-based estimates~\cite{tan2016LecNot, roberts2004general}.

The ergodicity theorem in~\cite{tan2016LecNot} states that, if a Markov chain $\mathcal{M}$ is aperiodic, irreducible, and reversible, then it converges to a unique stationary equilibrium distribution.
In other words, if the chain can reach any state from any other state (irreducibility), is not stuck in periodic cycles (aperiodicity), and is reversible (detailed balance condition~\cite{tan2016LecNot}).
Then the chain converges to the desired target distribution  with $ \mathcal{M} \sim \pi(\bm{x}, \bm{\theta} |  \bm{y})$.
In practice, one can inspect the trace $\pi(\bm{x}^{(k)}, \bm{\theta}^{(k)} |  \bm{y})$ for $k = 1, \dots, N$ and visually assess convergence and mixing properties of the chain to evaluate ergodicity.
The sampling methods used in this thesis possess proven ergodic properties, and we therefore refer the reader to the corresponding literature for further details.
Nevertheless, we will give a brief overview of the sampling algorithm used.

\subsection{Sampling from the marginal posterior}
As in Eq. \ref{eq:MTC}, when using the MTC method we sample from $\pi(\bm{\theta} |  \bm{y})$ first and then determine the full conditional $\pi(\bm{x} |  \bm{y})$ as in Eq. \ref{eq:fullCond}. To sample from $\pi(\bm{\theta} |  \bm{y})$, we use a Metropolis-within-Gibbs (MWG) sampler as described in~\cite{fox2016fast}.
We apply the MWG sample for the two-dimensional case only, with $\bm{\theta} = (\theta_1, \theta_2)$, where we perform a Metropolis step in the $\theta_1$ direction and a Gibbs step in the $\theta_2$ direction.
Ergodicity for this approach is proven in~\cite{roberts2006harris}.


The Metropolis-within-Gibbs algorithm begins with an initial guess $\bm{\theta}^{(t)}$ at $t=0$. We then propose a new sample $\theta_1 \sim q(\theta_1 |  \theta_1^{(t-1)})$, conditioned on the previous state, using a symmetric proposal distribution $q(\theta_1 |  \theta_1^{(t-1)}) = q(\theta_1^{(t-1)} |  \theta_1)$, which is a special case of the Metropolis-Hastings algorithm~\cite{roberts2006harris}.
We accept and set $\theta_1^{(t)} = \theta_1$ with the acceptance probability
\begin{align}
	\alpha(\theta_1 |  \theta_1^{(t-1)}) = \min \left\{ 1, \frac{\pi(\theta_1 |  \theta_2^{(t-1)}, \bm{y}) \, \cancel{q(\theta_1^{(t-1)} |  \theta_1)}}{\pi(\theta_1^{(t-1)} |  \theta_2^{(t-1)}, \bm{y}) \, \cancel{q(\theta_1 |  \theta_1^{(t-1)})}} \right\}
\end{align}
or reject and keep $\theta_1^{(t)} = \theta_1^{(t-1)}$, which we do by comparing $\alpha$ to a uniform random number $u \sim \mathcal{U}(0,1)$. 

Next, we perform a Gibbs step in the $\theta_2$ direction, where Gibbs sampling is again a special case of the Metropolis-Hastings algorithm with acceptance probability equal to one, and draw the next sample $\theta_2^{(t)} \sim \pi(\cdot |  \theta_1^{(t)}, \bm{y})$, conditioned on the current value $\theta_1^{(t)}$. 

We repeat this procedure $N^{\prime}$ times and ensure convergence independently of the initial sample (irreducibility) by discarding the initial $N_{\text{burn-in}}$ samples after a so-called burn-in period, resulting in a Markov chain of length $N = N^{\prime} - N_{\text{burn-in}}$.

\begin{algorithm}[!ht]
	\caption{Metropolis within Gibbs}
	\begin{algorithmic}[1]
		\STATE Initialise and suppose two dimensional vector \( \bm{\theta}^{(0)}  =( \theta_1^{(0)} , \theta_2^{(0)}  ) \)
		\FOR{ \( k = 1, \dots, N^{\prime} \)}
		\STATE Propose \( \theta_1 \sim q(\cdot   | \theta_1 ^{(t-1)}) = q(\theta_1 ^{(t-1)} |\cdot  ) \)
		\STATE Compute
		\[ \alpha( \theta_1  | \theta_1^{(t-1)}) = \min \left\{ 1, \frac{\pi(\theta_1  | \theta^{(t-1)}_2, \bm{y}) \cancel{q(\theta_1^{(t-1)} | \theta_1 ) } }{\pi(\theta_1^{(t-1)}| \theta_2^{(t-1)}, \bm{y}) \cancel{q(\theta_1 | \theta_1^{(t-1)})} } \right\} \]
		\STATE Draw $u \sim \mathcal{U}(0,1)$
		\IF{$\alpha \geq u$ }
		\STATE Accept and set \( \theta_1^{(t)} = \theta_1 \)
		\ELSE  
		\STATE Reject and keep \(\theta_1^{(t)} = \theta_1^{(t-1)} \)
		\ENDIF
		\STATE Draw \(\theta_2^{(t)} \sim  \pi( \cdot | \theta_1^{(t)} , \bm{y} )\) 
		\ENDFOR
		\STATE Output: $ \bm{\theta}^{(0)}, \dots,  \bm{\theta}^{(k)} , \dots,   \bm{\theta}^{(N)} \sim \pi(\bm{\theta}| \bm{y}) $
	\end{algorithmic}
	\label{alg:MwG}
\end{algorithm}


%\subsection{Draw a sample from a multivariate normal distribution}
%\label{subsec:RTO}
%As part of the MTC scheme, we only draw samples from the conditional distribution $\pi(\bm{x} |  \bm{\theta}, \bm{y})$ after sampling from the marginal posterior $\pi(\bm{\theta} |  \bm{y})$. For linear-Gaussian Bayesian hierarchical models, samples from the multivariate normal distribution $\pi(\bm{x} |  \bm{\theta}, \bm{y})$ can be efficiently generated using the Randomise-then-Optimise (RTO) method~\cite{bardsley2012mcmc}.
%
%The full conditional distribution can be rewritten as
%\begin{align}
%	\pi(\bm{x} |  \bm{y}, \bm{\theta}) &\propto \pi(\bm{y} |  \bm{x}, \bm{\theta}) \, \pi(\bm{x} |  \bm{\theta}) \\
%	&= \exp \left( -\left\lVert \hat{\bm{A}} \bm{x} - \hat{\bm{y}} \right\rVert^2 \right),
%\end{align}
%where
%\begin{align}
%	\label{eq:minimizer}
%	\hat{\bm{A}} = 
%	\begin{bmatrix}
%		\bm{\Sigma}^{-1/2}(\bm{\theta}) \bm{A} \\
%		\bm{Q}^{1/2}(\bm{\theta})
%	\end{bmatrix}, \quad 
%	\hat{\bm{y}} = 
%	\begin{bmatrix}
%		\bm{\Sigma}^{-1/2}(\bm{\theta}) \bm{y} \\
%		\bm{Q}^{1/2}(\bm{\theta}) \bm{\mu}
%	\end{bmatrix} \quad \text{\cite{bardsley2014randomize}}.
%\end{align}
%A sample $\bm{x}_i $ can be computed by minimising the following equation with respect to $\hat{\bm{x}}$ :
%\begin{align}
%	\bm{x}_i = \arg \min_{\hat{\bm{x}}} \lVert \hat{\bm{A}} \hat{\bm{x}} - ( \hat{\bm{y}} + \bm{b} ) \rVert^2 , \quad \bm{b} \sim \mathcal{N}(\bm{0}, \mathbf{I}) \, ,
%\end{align}
%where we add a randomised perturbation $\bm{b}$.
%Similar to Section~\ref{sec:regularise}, this expression can be rewritten as
%\begin{align}
%	\label{eq:RTO}
%	\left( \bm{A}^T \bm{\Sigma}^{-1}(\bm{\theta}) \bm{A} + \bm{Q}(\bm{\theta}) \right) \bm{x}_i = \bm{A}^T \bm{\Sigma}^{-1}(\bm{\theta}) \bm{y} + \bm{Q}(\bm{\theta}) \bm{\mu} + \bm{v}_1 + \bm{v}_2,
%\end{align}
%where the term $-\hat{\bm{A}}^T \bm{b}$ is decomposed as $\bm{v}_1 + \bm{v}_2$, with $\bm{v}_1 \sim \mathcal{N}(\bm{0}, \bm{A}^T \bm{\Sigma}^{-1}(\bm{\theta}) \bm{A})$ and $\bm{v}_2 \sim \mathcal{N}(\bm{0}, \bm{Q}(\bm{\theta}))$, representing independent Gaussian random variables~\cite{bardsley2012mcmc, fox2016fast}.
%
%If the Markov chain over the marginal posterior $\pi(\bm{\theta} |  \bm{y})$ is ergodic, and the conditional samples $\bm{x}^{(k)} \sim \pi(\bm{x}|   \bm{\theta}^{(k)}, \bm{y})$ are drawn independently, then the resulting joint chain $\{ (\bm{x}, \bm{\theta})^{(1)}, \dots, (\bm{x}, \bm{\theta})^{(N)} \} \sim \pi(\bm{x}, \bm{\theta} |  \bm{y})$ is also ergodic~\cite{acosta2014markov}.

\subsection{t-walk sampler as black box}
If the parameters $\bm{x}$ are functionally dependent on the hyper-parameters $\bm{\theta}$, i.e., $\bm{x} = \bm{x}(\bm{\theta})$, we can sample directly from the marginal posterior $\pi(\bm{\theta} | \bm{y})$ using the t-walk algorithm by Christen and Fox~\cite{christen2010general}. 
The t-walk is employed as a black-box sampler, requiring the specification of the number of samples, burn-in period, support region, and the target distribution. 
Convergence to the target distribution is guaranteed by construction of the algorithm.

\section{Numerical Approxiamtion Methods - Tensor Train}
\label{sec:tensortrain}
%\textcolor{red}{Explain how to find normalisation constant and say that due to aproxiamting the sqaure root we ensure posivyt later when squaring it}
Instead of sampling from a target distribution $\pi(\bm{x})$ we can approximate that distribution on a d-dimensional grid with far fewer function evaluation compared to sampling methods using a tensor train (TT) approximation $\tilde{\pi}(\bm{x}) \approx \pi(\bm{x})$, with  $\bm{x}\in \mathbb{R}^d$.
First, we provide a short overview of probability spaces and their associated measures, as a foundation for calculating marginal probability distributions from the tensor train format.
Note that we follow the notation of Cui et al. \cite{cui2022deep} to introduce this methodology.

Assume that the triple $(\Omega, \mathcal{F}, \mathbb{P})$ defines a probability space, where $\Omega$ denotes the complete sample space, $\mathcal{F}$ is a $\sigma$-algebra consisting of a collection of countable subsets $\{A_n\}_{n \in \mathbb{N}}$ with $A_n \subseteq \Omega$, and $\mathbb{P}$ is a probability measure defined on $\mathcal{F}$. The formal conditions for $\mathbb{P}$ to be a probability measure, and for $\mathcal{F}$ to be a $\sigma$-algebra over $\Omega$, are given in Appendix~\ref{ch:Mesure}.
We denote
\begin{align}
	\mathbb{P}(A) = \int_A \mathrm{d} \mathbb{P}
\end{align}
as the probability of an event $A \in \mathcal{F}$.
By applying the Radon-Nikodym theorem~\cite{kopp2004measintprob}, we can change variables
\begin{align}
	\mathbb{P}(A) = \int_A \frac{\mathrm{d} \mathbb{P}}{\mathrm{d}\bm{x}} \, \mathrm{d}\bm{x} = \int_A \pi(\bm{x}) \, \mathrm{d}\bm{x},
\end{align}
where $\mathrm{d}\bm{x}$ is a reference measure on the same probability space, commonly referred to as the Lebesgue measure. 
The Radon-Nikodym derivative $\frac{\mathrm{d} \mathbb{P}}{\mathrm{d}\bm{x}}$ of $\mathbb{P}$ with respect to $\bm{x}$ is often interpreted as the probability density function (PDF) $\pi(\bm{x})$. Thus, we say that $\mathbb{P}$ has a density $\pi(\bm{x})$ with respect to $\bm{x}$~\cite[Chapter 10]{simonnet1996measprob}.

Now, let $X: \Omega \longrightarrow \mathbb{R}^d$ be a $d$-dimensional random variable mapping from the probability space $(\Omega, \mathcal{F}, \mathbb{P})$ to the measurable space $(\mathbb{R}^d, \mathcal{X})$, where $\mathcal{X}$ is a collection of subsets in $\mathbb{R}^d$.
Then the associated PDF $\pi(\bm{x})$, is a joint density of $X$, induced by the probability measure on $\Omega$~\cite{VesaInvLect, kopp2004measintprob}.
As in~\cite{cui2022deep}, we can define the parameter space as the Cartesian product $\mathcal{X} = \mathcal{X}_1 \times \mathcal{X}_2 \times \dots \times \mathcal{X}_d$ with $ x_k \in \mathcal{X}_k \subseteq \mathbb{R}$ and $\bm{x} = ( x_1,\dots ,x_k,\dots,x_d )$.
The marginal density function for the $k$-th component is then given by
\begin{align}
	f_{X_k}(x_k) = \int_{\mathcal{X}_1} \cdots \int_{\mathcal{X}_d} \uplambda(\bm{x}) \, \pi(\bm{x}) \, \mathrm{d}x_1 \cdots \mathrm{d}x_{k-1} \, \mathrm{d}x_{k+1} \cdots \mathrm{d}x_d, \label{eq:margInt}
\end{align}
where we integrate over all dimensions except the $k$-th.
Here, we introduce a weight function $\uplambda(x)$, which can be useful for quadrature rules~\cite{davis2007methods}, to which~\cite{cui2022deep} refer to as a "product-form Lebesgue-measurable weighting function" and define it as
\begin{align*}
	\uplambda(\mathcal{X}) = \prod_{i = 1}^{d} \uplambda_i(\mathcal{X}_i), \quad \text{where} \quad \uplambda_i(\mathcal{X}_i) = \int_{\mathcal{X}_i} \uplambda_i(x_i) \, \mathrm{d}x_i. \label{eq:lebesgueWeight}
\end{align*}

In the tensor train (TT) format, the integral in Eq. \ref{eq:margInt} for the marginal probability can be computed at a low computational cost as $\pi(\bm{x})$ is approximated by
\begin{align*}
	\tilde{\pi}(\bm{x}) = 	\tilde{\pi}_1(x_1)  \tilde{\pi}_2(x_2)  \cdots \tilde{\pi}_d(x_d)  \in \mathbb{R},
\end{align*}
which is a sequence of matrix multiplications, with $\tilde{\pi}_k(x_k) \in \mathbb{R}^{r_{k-1} \times r_k}$ for a fixed grid point $\bm{x} = (x_1, \dots, x_d)$ on a $d$-dimensional discrete univariate grid over the parameter space $\mathcal{X}$ bounded by the outer ranks $r_0 = r_d = 1$.
We call $\tilde{\pi}_k \in \mathbb{R}^{r_{k-1} \times n \times r_k}$ a TT-core with ranks $r_{k-1}$ and $r_k$, representing each dimension on $n$ grid points.
This enables us to approximate $\pi(\mathcal{X})\approx \tilde{\pi}_1  \tilde{\pi}_2  \cdots \tilde{\pi}_d  \in \mathbb{R}^{d}$ using $2nr + (d-2)nr^2$ evaluation points, as illustrated in Figure~\ref{fig:TTfig}, instead of $n^d$ function evaluation.
Consequently, the marginal target distribution
\begin{align}
	\begin{split}
		f_{X_k}(x_k) = \frac{1}{z} \Big|\, 
		&\left( \int_{\mathbb{R}} \uplambda_1(x_1) \tilde{\pi}_1(x_1) \, \mathrm{d}x_1 \right) \cdots 
		\left( \int_{\mathbb{R}} \uplambda_{k-1}(x_{k-1}) \tilde{\pi}_{k-1}(x_{k-1}) \, \mathrm{d}x_{k-1} \right) \\
		&\quad \uplambda_k(x_k) \tilde{\pi}_k(x_k) \\
		& \left( \int_{\mathbb{R}} \uplambda_{k+1}(x_{k+1}) \tilde{\pi}_{k+1}(x_{k+1}) \, \mathrm{d}x_{k+1} \right) \cdots 
		\left( \int_{\mathbb{R}} \uplambda_d(x_d) \tilde{\pi}_d(x_d) \, \mathrm{d}x_d \right)
		\Big| 
	\end{split}
\end{align}
is computed by integrating over all TT cores except $\pi_k$, as in~\cite{dolgov2020approximation}, including a normalisation constant $z$~\cite{cui2022deep}.

\begin{figure}[ht!]
	\centering
\begin{subfigure}{\textwidth}
	\input{TTSchem.pdf_tex}
	\caption{}
\end{subfigure}
	\centering
\begin{subfigure}{\textwidth}
\begin{tikzpicture} 
	\node[rectnode] at (-5,0) (T1)    {$1 \times n \times r_1$};
	\node[rectnode] at (-2,0) (T2)    {$r_1 \times n \times r_{2}$};
	
	\node[rectnode] at (3,0) (Tn1)    {$r_{d-2} \times n \times r_{d-1} $};
	\node[rectnode] at (6.75,0) (Tn)    {$r_{d-1} \times n \times 1$};
	\draw[-, very thick] (T1.east) -- (T2.west); 
	\draw[-, very thick] (Tn.west) -- (Tn1.east); 
	\draw[-, mydotted, very thick] (T2.east) -- (Tn1.west);
	
	\node[align=center] at (-3.5,0.25) (R) {$r_1$};
	\node[align=center] at (5,0.25) (R) {$r_{d-1}$};	
\end{tikzpicture} 
	\caption{}
\end{subfigure}
\caption[Visualisation of a tensor train]{Here, we visualise the tensor train cores as two- and three-dimensional matrices. 
Each core has a length $n$, corresponding to the number of grid points in one dimension, and the cores are connected through ranks $r_k$. 
More specifically, a core $\tilde{\pi}_k$ has dimensions $r_{k-1} \times n \times r_k$, with outer ranks $r_0 = r_d = 1$.
Using the TT-format enables us to represent a $d$-dimensional grid with only $dnr^2$ evaluation points instead of $n^d$ grid points.
Figure~(a) is adapted from~\cite{fox2021grid}.}
\label{fig:TTfig}
\end{figure}


%%%%%% Squared and Basis Function %%%%%
In practice, tensor train approximations may suffer from numerical instability, in particular because it is not advantageous to approximate the target function $\pi(\bm{x})$ in e.g. the logarithmic space. 
Hence, Cui et al.~\cite{cui2022deep} approximate the square root of the probability density
\begin{align}
	\sqrt{\pi(\bm{x})} \approx \sqrt{\tilde{\pi}} = \bm{G}_1(x_1), \dots, \bm{G}_k(x_k), \dots, \bm{G}_d(x_d)\, ,
\end{align}
which ensures positivity.
Here, each TT-core is given by
\begin{equation}
	G^{(\alpha_{k-1},\alpha_k)}_k(x_k) = \sum_{i=1}^{n_k} \phi^{(i)}_k(x_k) \bm{A}_k[\alpha_{k-1}, i, \alpha_k], \quad k = 1, \dots, d,
\end{equation}
where $\bm{A}_k \in \mathbb{R}^{r_{k-1} \times n_k \times r_k}$ is the $k$-th coefficient tensor and $\{\phi^{(i)}_k(x_k)\}_{i=1}^{n_k}$ are the basis functions corresponding to the $k$-th coordinate.
The approximated density is written as:
\begin{align}
	\pi(\bm{x}) \approx \xi + (\sqrt{\tilde{\pi}})^2(\bm{x}),
\end{align}
where $\xi$ is a positive constant added according to the absolute error and the Lebesgue weighting, see Eq. \ref{eq:lebesgueWeight}, such that
\begin{align}
	0 \leq \xi \leq \frac{1}{\uplambda(\mathcal{X})} \lVert \sqrt{\tilde{\pi}} - \sqrt{\pi} \rVert_2^2.
\end{align}
This leads to the normalised target function
\begin{align}
	f_X(x)  \approx \frac{1}{z} \left( \uplambda(x) \xi  + \uplambda(x) \tilde{\pi}(x) \right),
\end{align}
where $z$ is the normalisation constant.
Given the tensor train approximation of $\sqrt{\pi}$, the marginal function $f_{X_k}(x_k)$ can be expressed as
\begin{align}
	\begin{split}
		f_{X_k}(x_k)  \approx \frac{1}{z} \Bigg(&\xi \prod_{i=1}^{k-1} \uplambda_i(\mathcal{X}_i) \prod_{i=k+1}^{d} \uplambda_i(\mathcal{X}_i) \\
		&+ \left( \int_{\mathbb{R}} \uplambda_1(x_1) \bm{G}_1^2(x_1)  \, \mathrm{d}x_1 \right) \cdots 
		\left( \int_{\mathbb{R}} \uplambda_{k-1}(x_{k-1}) \bm{G}_{k-1}^2(x_{k-1}) \, \mathrm{d}x_{k-1} \right) \\
		& \uplambda_k(x_k) \bm{G}_k^2(x_k)  \\
		&\left( \int_{\mathbb{R}} \uplambda_{k+1}(x_{k+1}) \bm{G}_{k+1}^2(x_{k+1})  \, \mathrm{d}x_{k+1} \right) \cdots 
		\left( \int_{\mathbb{R}} \uplambda_d(x_d) \bm{G}_d^2(x_d)  \, \mathrm{d}x_d \right) \Bigg).
	\end{split}
\end{align}




\subsection{Marginal Functions for quadrature}
To compute these marginals efficiently, one can use a procedure similar to left and right orthogonalisation of TT-cores~\cite{oseledets2011tensor}.
Cui et al. \cite{cui2022deep} referred to this backwards marginalisation, see Prop. \ref{prob:ForMarg}, to which I add the forward marginalisation, see Prob. \ref{prob:backMarg}. 
The backwards marginalisation provides us with the coefficient matrices $\bm{B}_k$, while the forward marginalisation gives the coefficient matrices $\bm{B}_{\text{pre}, k}$. 
These matrices enable the efficient evaluation of marginal functions since they integrate over the coordinates either left or right of the $k$-th dimension, as in \cite{cui2022deep}.
For this, we define the mass matrix $\bm{M}_k \in \mathbb{R}^{n_k \times n_k}$ as
\begin{equation}
	\bm{M}_k[i, j] = \int_{\mathcal{X}_k} \phi^{(i)}_k(x_k) \phi^{(j)}_k(x_k) \uplambda(x_k) \, \mathrm{d}x_k, \quad i, j = 1, \dots, n_k,
\end{equation}
where $\{\phi^{(i)}_k(x_k)\}_{i=1}^{n_k}$ denotes the set of basis functions for the $k$-th coordinate.
The proposition used to compute $\bm{B}_k$, stated in Proposition~\ref{prob:backMarg}, is adapted directly from~\cite{cui2022deep}.

\begin{prop}[Backwards Marginalisation as in~\cite{cui2022deep}]
	\label{prob:backMarg}
	Starting with the last coordinate $k = d$, we set $\bm{B}_d = \bm{A}_d$. The following procedure can be used to obtain the coefficient tensor $\bm{B}_{k-1} \in \mathbb{R}^{r_{k-2} \times n_{k-1} \times r_{k-1}}$, which we need for defining the marginal function $f_{X_k}(x_k)$:
	\begin{enumerate}
		\item Use the Cholesky decomposition of the mass matrix, $\bm{L}_k \bm{L}_k^\top = \bm{M}_k \in \mathbb{R}^{n_k \times n_k}$, to construct a tensor $\bm{C}_k \in \mathbb{R}^{r_{k-1} \times n_k \times r_k}$:
		\begin{equation}
			\bm{C}_k[\alpha_{k-1}, \tau, l_k] = \sum_{i=1}^{n_k} \bm{B}_k[\alpha_{k-1}, i, l_k] \bm{L}_k[i, \tau].
		\end{equation}
		\item Unfold $\bm{C}_k$ along the first coordinate and compute the thin QR decomposition, so that $\bm{C}_k^{(R)} \in \mathbb{R}^{r_{k-1} \times (n_k r_k)}$:
		\begin{equation}
			\bm{Q}_k \bm{R}_k = {(\bm{C}_k^{(R)})}^{\top}.
		\end{equation}
		\item Compute the new coefficient tensor:
		\begin{equation}
			\bm{B}_{k-1}[\alpha_{k-2}, i, l_{k-1}] = \sum_{\alpha_{k-1}=1}^{r_{k-1}} \bm{A}_{k-1}[\alpha_{k-2}, i, \alpha_{k-1}] \bm{R}_k[l_{k-1}, \alpha_{k-1}].
		\end{equation}
	\end{enumerate}
\end{prop}

\begin{prop}[Forward Marginalisation]
	\label{prob:ForMarg}
	Starting with the first coordinate $k = 1$, we set $\bm{B}_{\text{pre},1} = \bm{A}_1$. The following procedure can be used to obtain the coefficient tensor $\bm{B}_{\text{pre},k+1} \in \mathbb{R}^{r_{k} \times n_{k+1} \times r_{k+1}}$ for defining the marginal function $f_{X_k}(x_k)$:
	\begin{enumerate}
		\item Use the Cholesky decomposition of the mass matrix, $\bm{L}_k \bm{L}_k^\top = \bm{M}_k \in \mathbb{R}^{n_k \times n_k}$, to construct a tensor $\bm{C}_k \in \mathbb{R}^{r_{k-1} \times n_k \times r_k}$:
		\begin{equation}
			\bm{C}_{\text{pre},k}[\alpha_{k-1}, \tau, l_k] = \sum_{i=1}^{n_k} \bm{L}_k[i, \tau] \bm{B}_{\text{pre},k}[\alpha_{k-1}, i, l_k] .
		\end{equation}
		\item Unfold $\bm{C}_{pre,k}$ along the first coordinate and compute the thin QR decomposition, so that $\bm{C}_{\text{pre},k}^{(R)} \in \mathbb{R}^{(r_{k-1} n_k ) \times r_k}$:
		\begin{equation}
			\bm{Q}_{pre,k}\bm{R}_{\text{pre},k} = {(\bm{C}_{\text{pre},k}^{(R)})}.
		\end{equation}
		\item Compute the new coefficient tensor $\bm{B}_{\text{pre}, k+1} \in \mathbb{R}^{r_{k-1} \times n_k \times r_k} $:
		\begin{equation}
			\bm{B}_{\text{pre}, k+1}[l_{k+1}, i, \alpha_{k+1}] = \sum_{\alpha_{k}=1}^{r_{k}} \bm{R}_{\text{pre},k}[l_{k+1}, \alpha_{k}] \bm{A}_{k+1}[\alpha_{k}, i, \alpha_{k+1}] .
		\end{equation}
	\end{enumerate}
\end{prop}
After computing the coefficient tensors $\bm{B}_{\text{pre}, k+1}$ as in Prop.~\ref{prob:ForMarg} and $\bm{B}_{k+1}$ from Prop.~\ref{prob:backMarg}, the marginal PDF of $k$-th dimension can be expressed as
\begin{equation}
	f_{X_k}(x_k)  \approx \frac{1}{z} \left(\xi \prod_{i=1}^{k-1} \uplambda_i(X_i) \prod_{i=k+1}^{d} \uplambda_i(X_i) + \sum_{l_{k-1}=1}^{r_{k-1}} \sum_{l_k=1}^{r_k} \left(\sum_{i=1}^{n} \phi^{(i)}_k(x_k) \bm{D}_k[l_{k-1},i, l_k] \right)^2 \right) \uplambda_k(x_k),
\end{equation}
where $\bm{D}_k \in \mathbb{R}^{r_{k-1} \times n \times r_k}$ and $\bm{R}_{\text{pre},k-1}\in \mathbb{R}^{r_{k-1} \times r_{k-1}}$ and $\bm{B}_k \in \mathbb{R}^{r_{k-1} \times n \times r_k}$
\begin{equation}
	\bm{D}_k[l_{k-1},i,l_k] = \sum_{\alpha_{k-1}=1}^{r_{k-1}}  \bm{R}_{\text{pre},k-1}[l_{k-1}, \alpha_{k-1}] \bm{B}_k[\alpha_{k-1}, i, l_k].
\end{equation}

For the first dimension, $f_{X_1}(x_1)$ can be expressed as
\begin{equation}
	f_{X_1}(x_1)  \approx \frac{1}{z} \left(\xi \prod_{i=2}^{d} \uplambda_i(\mathcal{X}_i) + \sum_{l_1=1}^{r_1} \left(\sum_{i=1}^{n} \phi^{(i)}_1(x_1) \bm{D}_1[i, l_1] \right)^2 \right) \uplambda_1(x_1),
\end{equation}
where $\bm{D}_1[i, l_1] = \bm{B}_1[\alpha_0, i, l_1]$ and $\alpha_0 = 1$,
and similarly in the last dimension
\begin{equation}
	f_{X_d}(x_d)  \approx \frac{1}{z} \left(\xi \prod_{i=1}^{d-1} \uplambda_i(\mathcal{X}_i) + \sum_{l_{n-1}=1}^{r_{d-1}} \left(\sum_{i=1}^{n} \phi^{(i)}_1(x_1) \bm{D}_d[l_{n-1},i] \right)^2 \right) \uplambda_d(x_d),
\end{equation}
where $\bm{D}_d[l_{n-1},i] = \bm{B}_{\text{pre},d}[l_{n-1}, i, \alpha_{n+1}]$ and $\alpha_{d+1} = 1$.
Note that we calculate the normalisation numerically within the process of finding the marginals so that $\sum f_{X_k}(x_k) = 1$.

\subsection{Sampling from the TT approximation -- Inverse Rosenblatt Transform}
%\chapter{Results}
%\label{ch:res}
%
%\section{Simulate data}
%
%
%\section{Develop a hierarchical Bayesian model}
%\label{sec:applBay}
%
%\subsection{prior modelling}
%
%\section{affine map with posterior ozone}
%\subsection{sampling vs TT}
%
%\section{posterior ozone vs regularisation}
%\label{sec:applReg}
%\subsection{sampling vs TT}
%
%\section{posterior pressure and temperature}
%\subsection{sampling vs TT}
%


\section{Affine Map}
\label{sec:affine}
The forward map, which we introduce in Ch. \ref{ch:formodel}, poses a weakly non-linear forward problem, which we could tackle by treating the problem as a linear problem and then iteratively updating the non-linear part after each parameter sample.
Instead, we approximate the non-linear model using an affine map $ \bm{M}:\bm{A}_L \bm{x} \rightarrow \bm{A}_{NL}\bm{x}$, which approximates the non-linear model using the linear model.
Here we give a brief introduction to affine maps and present our approach to calculating the affine map deterministically. 
Alternatively, one can also determine this map using other methods, e.g. machine learning methods \cite{}.


An affine map is any linear map between two vector spaces or affine spaces, where an affine space does not need to preserve a zero origin, see~\cite[Def. 2.3.1]{berger2009geometry}.
In other words, an affine map does not need to map to the origin of the associated vector space or be a linear map on vector spaces, including a translation, or, in the words of my supervisor, C. F., an affine map is a Taylor series of first order.
For more information on affine spaces and maps, we refer to the books \cite{berger2009geometry, katsumi1994affine}
\begin{figure}[ht!]
	\centering
	\begin{tikzpicture}
		\node[rectnode] at (2,-4) (NL)    {$\bm{V}$};
		\node[rectnode] at (-2,-4) (L)    {$\bm{W}$};
		\draw[<-, very thick] (NL.west) -- (L.east); 
		\node[align=center] at (-5.5,-4) (f3) {linear forward model};
		\node[align=center] at (5.5,-4) (f4) {non-linear forward model};
		\node[align=center] at (0,-5) (f5) {$\bm{A}_{NL} \bm{x}  \approx \bm{M A}_L  \bm{x} = \bm{A x} $ };
		\node[align=center] at (0,-4) (f5) {affine Map \\ $\bm{M}$};
	\end{tikzpicture}
	\caption[Schematics of the affine map]{This Figure shows the schematic representation of the affine map $\bm{M}$, which approximates the non-linear forward model from the linear forward model. Here, $V$ contains values produced by the linear forward model, and $W$ contains the corresponding values from the non-linear forward model. Both $V$ and $W$ are affine subspaces over the same field. The affine map $\bm{M}$ projects elements from the linear forward model space $V$ onto their counterparts in the non-linear forward model space $W$.
	}
\end{figure}


Consequently, to map between the linear and non-linear forward map, we generate two affine subspaces $V$ and $W$ over the same field.
Assume we have noise free data vector $A_{NL}\bm{x} \in \mathbb{R}^m$, then the subspace associated with the linear forward model is \begin{align}
	\bm{W} = \begin{bmatrix}
		\vert&   &  \vert & & \vert \\
		\bm{A}_{L} \bm{x}^{(1)} &  \cdots& \bm{A}_{L} \bm{x}^{(j)} &  \cdots & \bm{A}_{L} \bm{x}^{(m)} \\
		\vert&   &  \vert & & \vert 
	\end{bmatrix}\in \mathbb{R}^{m \times m}
\end{align} and with the non-linear forward model is 
\begin{align}
	\bm{V} = \begin{bmatrix}
		\vert&   &  \vert & & \vert \\
		\bm{A}_{NL}\bm{x}^{(1)} &  \cdots& \bm{A}_{NL}\bm{x}^{(j)} &  \cdots & \bm{A}_{NL} \bm{x}^{(m)}  \\
		\vert&   &  \vert & & \vert 
	\end{bmatrix} = 
	\begin{bmatrix}
		\rule[.5ex]{1em}{0.4pt} & v_1 &	\rule[.25ex]{1em}{0.4pt}\\
		&\vdots  & \\
		\rule[.5ex]{1em}{0.4pt} &v_j &	\rule[.25ex]{1em}{0.4pt}\\
		&\vdots  & \\
		\rule[.5ex]{1em}{0.4pt}& v_m&	\rule[.25ex]{1em}{0.4pt}
	\end{bmatrix} \in \mathbb{R}^{m \times m} \, ,
\end{align}
Here $\big\{  \bm{x}^{(1)} , \dots, \bm{x}^{(m)} \big\}$ are $m$ different parameters.

Then we find the affine map 
\begin{align}
	\bm{V}\bm{W}^{-1} = \bm{M} = \begin{bmatrix}
		\text{---} & r_0 &   \text{---}  \\
		&  \vdots  & \\
		\text{---}& r_j &  \text{---} \\
		&  \vdots  & \\
		\text{---} & r_m &   \text{---}
	\end{bmatrix} \, \in \mathbb{R}^{m \times m} ,
\end{align}
row wise by solving $v_j = r_j \bm{W}$ for $ r_j $, so that $ \bm{A} = \bm{M} \bm{A}_{NL} \approx \bm{A}_{NL} $.
Alternately one could also compute $\bm{M}$ using the inverse $\bm{W}^{-1}$.

\section{Regularisation}
\label{sec:regularise}
As mentioned in the introduction, the currently most used method to analyse any data in atmospheric physics is regularisation.
Since we want to show that our methods are computationally comparable if not faster, and provide more information than regularisation, we choose a regulariser closest to our linear-Gaussian Bayesian framework, see section \ref{sec:BayModel}.

The Tikhonov approach in provides one solution $\bm{x}_{\lambda}$ that minimises both the data misfit norm
\begin{align}
	\left\lVert \bm{y} - \bm{A} \bm{x} \right\rVert
\end{align} and a regularisation semi-norm
\begin{align}
	\lambda \left\lVert \bm{T} \bm{x} \right\rVert
\end{align}, as described in~\cite{fox2016fast}, with a linear forward model matrix $\bm{A}$, the data $\bm{y}$ and a regularisation operator $\bm{T}$ and the regularisation parameter $\lambda > 0$ which penalise $\bm{x}$ accordingly.
For a fixed $\lambda$, the regularised solution
\begin{align}
	\bm{x}_{\lambda} = \underset{\bm{x}}{\mathrm{arg\,min}} \left\lVert \bm{y} - \bm{A} \bm{x} \right\rVert^2 + \lambda \left\lVert \bm{T} \bm{x} \right\rVert^2
\end{align}
is obtained by taking the derivative with respect to $\bm{x}$ of the objective function:
\begin{align}
	& & \nabla_{\bm{x}} \left\{ (\bm{y} - \bm{A} \bm{x})^T (\bm{y} - \bm{A} \bm{x}) + \lambda \bm{x}^T \bm{T}^T \bm{T} \bm{x} \right\} &= 0 \\
	&\iff & \nabla_{\bm{x}} \left\{ \bm{y}^T \bm{y} + \bm{x}^T \bm{A}^T \bm{A} \bm{x} - 2 \bm{y}^T \bm{A} \bm{x} + \lambda \bm{x}^T \bm{T}^T \bm{T} \bm{x} \right\} &= 0 \\
	&\iff & 2 \bm{A}^T \bm{A} \bm{x} - 2 \bm{A}^T \bm{y} + 2 \lambda \bm{T}^T \bm{T} \bm{x} &= 0,
\end{align}
or equivalently the "regularised normal equations" $\bm{A}^T \bm{y} = \bm{A}^T \bm{A} \bm{x} + \lambda \bm{T}^T \bm{T} \bm{x}$ \cite{Hansen2001LCurve}.
Solving this equation yields the regularised solution
\begin{align}
	\bm{x}_{\lambda} = (\bm{A}^T \bm{A} + \lambda \bm{L})^{-1} \bm{A}^T \bm{y} \, , \label{eq:regSol}
\end{align}
where we define $\bm{L} := \bm{T}^T \bm{T}$, which typically represents a discrete matrix approximation of a differential operator choice~\cite{tan2016LecNot}.

In practice, $\bm{x}_{\lambda}$ is computed for a range of $\lambda$-values and evaluated based on the trade-off between the data misfit and the regularisation norm. The optimal value of $\lambda$ is often chosen as the point of maximum curvature on the so-called L-curve~\cite{hansen1993use}, which we plot in Fig.~\ref{fig:LCurve}.
Additionally one can think about it .... in \cite{fox2016fast}
\cite[Fig. 2.13]{SANTOSH202265}

