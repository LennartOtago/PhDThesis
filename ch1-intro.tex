\chapter{Introduction}
Here, we briefly describe the currently used standard to retrieve atmospheric trace gas concentrations from limb-sounding measurements and what motivates us to employ a hierarchical Bayesian framework addressing this inverse problem.
We explain how our approach contributes to and improves upon existing methods.
Lastly, we provide the reader with the thesis structure.


\section{Motivation}
The only currently operating ozone limb sounder, the Microwave Limb Sounder (MLS) on NASA's Aura satellite, is gradually drifting from its orbit and scheduled to be phased out by 2026~\cite{Bryan2024NASA}.
A group led by Harald Schwefel has proposed an alternative approach to fill this observational gap using a much smaller platform such as a 6U CubeSat (roughly $28\text{cm} \times 15\text{cm} \times 9\text{cm}$)~\cite{ustin2024current}. 
The proposed system includes a disk-shaped resonator targeting a narrow frequency band and converting the thermal radiation emitted by ozone molecules from the terahertz region to the optical domain~\cite{Suresh25,Sedlmeir14}. 
This frequency conversion offers a cost-effective and energy-efficient solution, as it circumvents the need for large, energy-hungry cooling devices that are traditionally required to capture terahertz signals. 
Instead, signal acquisition in the optical domain can be implemented by using compact, cheap, and low-power photonic technologies.

Currently, the inverse problem to retrieve any trace gas from limb-sounding data is approached by the atmospheric physics community using optimisation and regularisation techniques developed in the 1970s~\cite{rodgers1976retrieval, NASA2022MLSv5}.
These methods focus on finding the ``best fit to data but not the best fit to parameters''~\cite{tan2016LecNot}.
Instead, we employ a hierarchically structured Bayesian framework to provide a distribution of ozone profiles, which represents a range of feasible solutions to some given data.
This probabilistic approach allows us to determine meaningful estimates and uncertainties of parameters.

\section{Research Gap and Contribution}

As already mentioned, currently the MLS retrieval algorithm \cite{livesey2006retrieval} is based on the ``optimal estimation'' method from \cite{rodgers1976retrieval}.
This approach provides a point estimate by fitting parameters to some data and iteratively minimising a squared residual norm, penalised against a chosen regularisation.
This does not provide comprehensive information about the parameters' underlying correlation structures can lead to unphysical results, e.g. negative ozone concentration values \cite{MLSdata}, and biased solutions, where the bias is then removed based on empirical decisions \cite{livesey2008ozonecarbonmono, Froidevaux2008snrozone}.
Errors are provided by a local derivative of the forward map at the optimal solution, which is inherently highly sensitive to that specific point in the parameter space.
Furthermore, these regularisation methods condition on external point estimates of other parameters, such as temperature or pressure \cite{livesey2006retrieval}.
%Recent machine learning efforts do not include noise as a retrieval hyper-parameter condition on one single noise value (point estimate) in the developed model, which is trained for about one month, and .
%Additionally they compare to a ``ground truth'' provided by the previously described optimal estimation approach \cite{werner2023machlearn, bojkov2008NeuralNet}.
%Recent machine learning approaches suffer from related limitations. These models are typically trained over short periods (e.g., one month), condition on a single fixed noise value, and do not treat noise as a retrieval hyper-parameter. Moreover, they are benchmarked against a “ground truth” derived from the same optimal estimation method described above~\cite{werner2023machlearn, bojkov2008NeuralNet}.

In Bayesian hierarchial modelling inknown parameter are treated as random variables and are included if computatinlly feasinel within the incersion prcess.
This means that noise is controlled by a hyperparameter and included in the model as well as other hypeparaemntes and ozone as a paramtere.
Since we deal with random varaibel described by distributions we provide posterior distribution of variabel which all present feasible solution to the data instead of on ``optimal'' solution.

Naturally, noise (hyper-parameter) is a random process and follows a probability distribution, which we assume to know.
According to that noise, as well as some other probability distributions, we can formulate explicit functions over hyper-parameters and parameters (e.g. ozone concentrations).\textcolor{red}{why is this here -- not clear to me}
By incorporating hyper-parameters, e.g. measurement noise and smoothness of the ozone profile, in the modelling and inversion process, we are able to provide errors and a range of feasible paramters (posterior distribution) given some data, instead of one ``optimal'' solution.\textcolor{red}{ what kind of functions - -presently I can't work out what this says}
This approach is known as hierarchical Bayesian modelling.\textcolor{red}{  OH, your trying to say that a hyperparametr controls something about the distribution of the noise, not that the noise is a hyperparameter, which is how this sentence reads.}
Livesey et al.~\cite{livesey2006retrieval} report ``unexpected spectrally correlate noise'' on the MLS aura, so here is another real reason why one should include noise in the model.

We utilise the marginal and then conditional (MTC) method~\cite{fox2016fast} 
and set up a hierarchical Bayesian model based on the radiative transfer equation (RTE).
This is a relatively new method within the Bayesian community, and is the first application to a forward model based on the radiative transfer equation (RTE), to the best of our knowledge.
We simulate some data after given a brief assesment of one can measure most effectivley.
First we neglet the non-linearity in the RTE and employ a linear-Gaussian hierarchical Bayesian, which we compare to a regularisation approach.
Within seconds we can evaluate the marginal posterior distribution over the hyper-parameters and the conditional posterior distribution over the parameters.
Since the RTE is weakly non-linear, we use these resuts to find an affine map that approximates the non-linear forward model, which seems to be another novelty in the field of atmospheric remote sensing.
Finally we extend our model to jointly provide posterior pressure, temperaure and ozone samples.
Additionally, we extend the MTC method to tackle this inverse problem by jointly inferring pressure, temperature and ozone profiles given one set of measurements.

Moreover, instead of using sampling algorithms to characterise those posterior distributions the approximate posterior distributions directly utilising the tensor-train (TT) format. 
This allows to generate independent samples from a TT approximation via the inverse Rosenblatt transform (IRT) with far fewer function evaluations compared to conventional samplers.
Moreover, the TT format allows us to calculate marginal distributions of each hyper-parameter and evaluate integrals via quadrature without any sampling.


and first the treat the 
to solve this inverse problem and employ a linear-Gaussian hierarchical Bayesian framework.
Within seconds we can evaluate the marginal posterior distribution over the hyper-parameters and the conditional posterior distribution over the parameters. \textcolor{red}{this does not clearly say that you treat a nonliner problem later.}
This is a relatively new method within the Bayesian community, and is the first application to a forward model based on the radiative transfer equation (RTE), to the best of our knowledge. \textcolor{red}{already too many "we". Try something like : the sthe application in this thesis is ... the fist, to the besy of our knowledge.}

\textcolor{red}{likewise. Besides, this sentence is not true. You know about the Approximation and Sampling paper.}
\section{Thesis Structure}
In Chapter~\ref{ch:background}, we give a brief overview of the key methods used along with references for further reading.
Chapter~\ref{ch:formodel} introduces the simplified forward model based on the RTE, and discusses strategies for measuring effectively.
Based on this, we simulate noisy data for an idealised limb sounder within a simplified atmosphere.
Then, in Chapter~\ref{ch:res}, we construct our linear-Gaussian hierarchical Bayesian model and discuss some prior modelling choices.
Given the simulated data, we apply the MTC method to provide posterior distributions of our Bayesian framework based on the linearised RTE to then approximate the non-linear forward model with an affine map. 
We compare a regularisation solution with the posterior distributions of the approximated linear forward model and a ground truth ozone profile.
Further, we extend the previously built Bayesian model to include hyper-parameters corresponding to pressure and temperature and touch on some prior modelling choices. 
Again, using the MTC method we jointly estimate posterior ozone, pressure and temperature profiles and highlight important aspects for improving the effectiveness and stability of TT approximations.
Lastly, we summarise and discuss our results and provide an outlook for future work, see Chapter~\ref{ch:Concl}.

