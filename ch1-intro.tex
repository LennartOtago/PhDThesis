\chapter{Introduction}

Here we briefly describe the standard currently used to retrieve atmospheric trace gas concentrations, e.g. ozone concentration, from limb-sounding measurements and what motivates us to employ a hierarchical Bayesian framework to address this inverse problem.
We explain how our approach contributes to and improves upon existing methods.
Lastly, we provide the reader with the thesis structure.


\section{Motivation}
Presently, the only operating ozone limb sounder is the Microwave Limb Sounder (MLS) on NASA's Aura satellite.
This satellite is gradually drifting away from its orbit and scheduled to be phased out by 2026~\cite{Bryan2024NASA}.
A group led by Harald Schwefel has proposed an alternative approach to fill this observational gap using a much smaller platform such as a 6U CubeSat (roughly $30\text{cm} \times 15\text{cm} \times 10\text{cm}$)~\cite{ustin2024current}. 
The proposed system includes a disk-shaped resonator targeting a narrow frequency band and converting the thermal radiation emitted by ozone molecules from the terahertz region to the optical domain~\cite{Suresh25,Sedlmeir14}. 
This frequency conversion offers a cost-effective and energy-efficient solution as it avoids the need for large, energy-hungry cooling devices that are traditionally required to capture terahertz signals. 
Instead, signal acquisition in the optical domain can be implemented by using compact, cheap, and low-power photonic technologies.

Currently, the inverse problem to retrieve any trace gas from limb-sounding data is approached by the atmospheric physics community using optimisation and regularisation techniques developed in the 1970s~\cite{rodgers1976retrieval, NASA2022MLSv5}.
These methods focus on finding the ``best fit to data but not the best fit to parameters''~\cite{tan2016LecNot}.
Instead, we employ a hierarchically structured Bayesian framework to provide a distribution of ozone profiles, which represents multiple possible solutions according to some given data.
This probabilistic approach allows us to determine meaningful estimates and uncertainties of parameters.

\section{Research Gap and Contribution}

As already mentioned, currently the MLS retrieval algorithm~\cite{livesey2006retrieval} is based on the ``optimal estimation'' method from~\cite{rodgers1976retrieval}.
This approach provides a point estimate by fitting parameters to some data and iteratively minimising a squared residual norm, penalised against a chosen regularisation.
However, this does not provide comprehensive information about the parameters' underlying correlation structures and can lead to unphysical results, e.g.~negative ozone concentration values~\cite{MLSdata}.
Additionally, the solutions may be biased and the bias is then removed based on empirical decisions~\cite{livesey2008ozonecarbonmono, Froidevaux2008snrozone}.
Errors are provided by a local derivative of the forward map at the optimal solution, which is inherently highly sensitive to that specific point in the parameter space.
Furthermore, these regularisation methods condition on external point estimates of other parameters such as temperature or pressure~\cite{livesey2006retrieval}.
%Recent machine learning efforts do not include noise as a retrieval hyper-parameter condition on one single noise value (point estimate) in the developed model, which is trained for about one month, and .
%Additionally they compare to a ``ground truth'' provided by the previously described optimal estimation approach \cite{werner2023machlearn, bojkov2008NeuralNet}.
%Recent machine learning approaches suffer from related limitations. These models are typically trained over short periods (e.g., one month), condition on a single fixed noise value, and do not treat noise as a retrieval hyper-parameter. Moreover, they are benchmarked against a “ground truth” derived from the same optimal estimation method described above~\cite{werner2023machlearn, bojkov2008NeuralNet}.

In the presence of measurement noise any observation process is a random process.
Consequently, in Bayesian modelling all unknown parameters are treated as random variables~\cite{kaipio2005statinv}.
Further, a hierarchically ordered model incorporates unknown hyper-parameters, which for example,~control the noise of the data and the smoothness of the ozone profile.
Therefore, it is able to model conditional dependencies between parameters and hyper-parameters.
For some given data, the posterior probability distribution of this hierarchical Bayesian framework provides a range of feasible unbiased solutions and hence true errors.
Livesey et al.~\cite{livesey2006retrieval} report ``unexpected spectrally correlated noise'' on the MLS Aura, so here is another real reason why one should include noise in the model.

In this thesis, the marginal and then conditional (MTC) method~\cite{fox2016fast} is utilised and a hierarchical Bayesian model based on the radiative transfer equation (RTE) is developed.
First, the non-linearity of the RTE is neglected and a linear-Gaussian hierarchical Bayesian model is employed to provide a posterior distribution of ozone profiles.
Since the RTE is weakly non-linear we use the previously obtained results to find an affine map that approximates the non-linear forward model.
This appears to be a novelty in the field of atmospheric remote sensing.
Lastly, the hierarchical Bayesian framework is extended to jointly infer ozone, pressure and temperature.
The MTC scheme is a relatively new method within the Bayesian community, and we are the first, to the best of our knowledge, to apply it to a forward model based on the RTE and to jointly provide posterior ozone, pressure, and temperature profiles.

Usually sampling algorithms are used to characterise the posterior probability distribution of a hierarchical Bayesian model. Instead, we approximate the posterior distribution directly utilising the tensor-train (TT) format~\cite{cui2022deep}. 
This allows us to generate independent samples from a TT approximation via a scheme similar to the inverse Rosenblatt transform (IRT)~\cite{dolgov2020approximation} with far fewer function evaluations compared to conventional samplers.
Further, in the TT format one can calculate marginal probability distributions of each hyper-parameter and evaluate integrals via quadrature without any sampling.


%and first the treat the 
%to solve this inverse problem and employ a linear-Gaussian hierarchical Bayesian framework.
%Within seconds we can evaluate the marginal posterior distribution over the hyper-parameters and the conditional posterior distribution over the parameters. \textcolor{red}{this does not clearly say that you treat a nonliner problem later.}
%This is a relatively new method within the Bayesian community, and is the first application to a forward model based on the radiative transfer equation (RTE), to the best of our knowledge. \textcolor{red}{already too many "we". Try something like : the sthe application in this thesis is ... the fist, to the besy of our knowledge.}
%
%
%This means that noise is controlled by a hyperparameter and included in the model as well as other hypeparaemntes and ozone as a paramtere.
%Since we deal with random varaibel described by distributions we provide posterior distribution of variabel which all present feasible solution to the data instead of on ``optimal'' solution.
%
%Naturally, noise (hyper-parameter) is a random process and follows a probability distribution, which we assume to know.
%According to that noise, as well as some other probability distributions, we can formulate explicit functions over hyper-parameters and parameters (e.g. ozone concentrations).\textcolor{red}{why is this here -- not clear to me}
%By incorporating hyper-parameters, e.g. measurement noise and smoothness of the ozone profile, in the modelling and inversion process, we are able to provide errors and a range of feasible paramters (posterior distribution) given some data, instead of one ``optimal'' solution.\textcolor{red}{ what kind of functions - -presently I can't work out what this says}
%This approach is known as hierarchical Bayesian modelling.\textcolor{red}{  OH, your trying to say that a hyperparametr controls something about the distribution of the noise, not that the noise is a hyperparameter, which is how this sentence reads.}
%\textcolor{red}{likewise. Besides, this sentence is not true. You know about the Approximation and Sampling paper.}
\section{Thesis Structure}
\begin{itemize}
	\item In Chapter~\ref{ch:background}, a brief introduction to hierarchical Bayesian modelling, sample-based estimates, TT approximations and regularisation approaches is given. We provide some background information along with references for further reading.
	\item Chapter~\ref{ch:formodel} introduces the simplified forward model based on the RTE following the Michelson interferometer for passive atmospheric sounding (MIPAS).
	We simulate data based on a ground truth and explain the process of doing so.
	Five different ways to acquire data are tested, and a singular value decomposition (SVD) of the forward model is performed to assess its information content.
	Given the SVD of the test cases, the most effective measurement method is determined.
	\item In Chapter~\ref{ch:LinVsReg}, the problem is treated as a linear inverse problem by neglecting the absorption term in the RTE. Some prior modelling aspects are discussed, and a linear-Gaussian hierarchical Bayesian model is established.
	Within the MTC scheme the marginal posterior over the hyper-parameters is evaluated first and then the full conditional posterior for ozone.
	Here, the marginal posterior is approximated in the TT format and the TT approximation is compared to sample-based results.
	Then posterior ozone profiles and a regularised estimate are compared to the ground truth.
	The chosen regularisation framework is the closest equivalent to the hierarchical Bayesian model in this Chapter.
	\item In Chapter~\ref{ch:affine}, the previously obtained posterior ozone profiles are used to find an affine map that approximates the non-linear RTE. The hierarchical Bayesian model from Chapter~\ref{ch:LinVsReg}, but with the approximate forward model, is used to obtain the posterior distribution of ozone profiles via the MTC scheme.
	\item In Chapter~\ref{ch:FullBay}, we extend the hierarchical Bayesian model to include pressure and temperature-related hyper-parameters.
	Prior modelling choices are discussed and the MTC method is utilised to obtain the marginal posterior over the hyper-parameters first and then the full conditional posterior of ozone.
	The reader is guided through the process of setting up an efficient 18-dimensional TT approximation of the marginal posterior.
	Some important aspects for improving the effectiveness and stability of TT approximations are highlighted.
	The TT approximation is compared to the results of the t-walk sampler on the marginal posterior.
	Posterior pressure and temperature profiles are obtained by sampling from the marginal posterior.
	Ozone samples of the full conditional posterior are drawn via the randomise then optimise method.
	\item In Chapter~\ref{ch:Concl}, some of the key differences between a regularisation approach and a hierarchical Bayesian approach are pointed out. Further, the advantages and disadvantages of TT approximations compared to conventional sampling methods are discussed.
	Lastly, we situate our results in the context of atmospheric physics and provide an outlook for future work.
\end{itemize}
All programming and analysis in this thesis are done in Python, and the reported computation times are taken on a MacBook Pro from 2019 with a 2.4 GHz quad-core Intel i5 processor.