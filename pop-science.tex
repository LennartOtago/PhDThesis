%this is the popiular science abstract
I love ice cream and when I stand in front of ice cream shop I have a hard time deciding my composition of ice cream meal.
There are too many options, two scoops three scoops, how may flavours does that include, cone or cup and the most important which flavours.
In this abstract I try to guide you how I decide among all those options and what my reasons behind are, so that you can compose you favourite icream cup.


Usually the first thing you decide is whether you like one, two, or three scopes in a cup or cone, which gives me already six options.
In terms of statistics we ask how many parameters we have to determine.
We decide on that depending on some prior knowledge e.g. how many scoops fit in a cup or cone and maybe even how hungry we are on that specific day or how the weather is like hot or cold.

Picking ice cream flavours does sound trivial; I assure you it is not.
We have to take into account our previus decision and based on this we know how the balls of iceream will be arranged.
This gives us a certain environment in which the ice cream is put and how decides how the ice cream is correlated.
Imagine having a cone the ice cream ball are arranged vertical in a chain and have one or maximum two neighbours.
If eating out a cup we assume that all ice cream flavours are a clique and all neighbours of each other.
If you ask me for my humble opinion this is a very important information, because there certain flavours you just don't want have next to each other, which I leave to your imagination and taste.
So depending on the ice cream setup, one could say field, we choose a combination of neighbours.
Depending on the person we choose different, one might like to combine fruits and chocolate or sweet and sour or just likes to stay with what is know and go for all kinds of chocolate ice cream on offer.
So we start choosing one ice cream flavour.

Given the first flavour we like to choose the next one.
If the ice cream shop is pretty nice we are allowed to try a bit and get the taste of that potential next flavour.
If we like it we will probably take it and if it matches with our expectations and expected composition.
If we are unsure we ask a friend and based on his recomendation we still take this favour eventhoug we are not 100 \% convinced.
Going through step by step until we completed a cup or cone.

At the end we end up with a cup or cone of ice cream depedning on our prior knowledge.
This cup has the most potential to satisfy our taste and to make our ice cream experience most enjoyable.
In this Ph.D. thesis it is all about how to compose such a ice cream portion and if we eat it out if a cup or cone or maybe something completely different.


% Do I take fruit but then I should take somnething frutixy next or oppisit

% and if I can try the next one do i take it or not 
% and how many tries di I have

% I have to take decision and cant try everything so I have to go with my best guess
% my best prior knwokledge because i have tried Icecream before and I knwo how it should taste but cant be cetrein if i taste the same
% depdening on gthe indgridunnes
% strawberry icecream taste different in summer
% maybe the recepie changed or the icream lady has a bad day mixed uop the recepies..

% so hard decision but at the end I go with something i most likelxy like

