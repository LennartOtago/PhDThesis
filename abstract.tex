\vspace{-1ex}
In this thesis, we develop a hierarchical Bayesian model based on the radiative transfer equation for an idealised atmospheric limb-sounder targeting ozone, as described in~\cite{mipas2000handbook}.
First, we linearise the radiative transfer equation by neglecting absorption.
To ensure effective measurements, we briefly assess the informativity of different measurement approaches via a singular value decomposition of the linear forward model.
According to these test results, we simulate data based on a ground truth.
To provide posterior distributions of hyper-parameters and ozone profiles a linear-Gaussian hierarchical Bayesian framework is established and the marginal and then conditional scheme~\cite{fox2016fast} is utilised.
A regularised estimate is compared to the posterior distribution of ozone profiles.
Then the non-linear forward model is approximated with an affine map.
The previously obtained hierarchical Bayesian framework is extended, and the marginal and then conditional scheme is applied to jointly infer posterior ozone, pressure and temperature.

Tensor-train function representations are applied to approximate high-dimensional posterior probability distributions~\cite{cui2022deep, dolgov2020approximation}.
This enables us to generate samples from the target distribution with far fewer function evaluations compared to the t-walk sampling algorithm~\cite{christen2010general}.
Tensor-train methods require a predefined grid and a ``normalisation constant'' so that function outputs are within computer precision, but once defined, they reduce the function evaluations per independent sample significantly.
Another advantage of the tensor-train format is that marginal probability distributions, useful for characterisation of integrals via quadrature, can be calculated at a low computational cost, without any sampling.
To further improve tensor-train methods, we suggest future work should focus on lowering tensor ranks, calculating ``normalisation constants'' to avoid numerical issues and reducing correlation structures between parameters automatically, all of which we currently have to do by exploratory analysis.
Additionally, choosing accurate interpolation schemes between grid points is crucial to improving the effectiveness of the approximation.

Our results show that a hierarchical Bayesian approach, which quantifies posterior mean and variance of the parameter (ozone), provides more information than a regularisation approach at comparable computational time.
In regions where the signal strength is low and the data is noise-dominated, we cannot recover ozone structures from the ground truth.
When including pressure and temperature describing hyper-parameters within our hierarchical Bayesian model, we find a strong correlation between ozone and pressure, whereas the model and data are uninformative about temperature.
For future work, we recommend developing a more physically informed parametrised model for ozone within the atmosphere, incorporating atmospheric chemistry and other important processes.