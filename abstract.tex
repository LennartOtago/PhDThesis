In this thesis, we develop a hierarchical Bayesian model for an atmospheric limb-sounder targeting ozone, as described in~\cite{mipas2000handbook}.
To ensure effective measurements, we briefly assess the informativity of the forward model for different measurement approaches and adapt the data collection accordingly.

Following~\cite{fox2016fast}, we utilise the marginal and then conditional scheme to provide posterior distributions of hyper-parameters and ozone profiles, and compare with a Tikhonov regularisation approach.
Further, we extend our model and the marginal and then conditional scheme to jointly infer posterior pressure, temperature and ozone.

The main contribution of this work is that we introduce a novel approach to approximate high-dimensional posterior distributions in the tensor-train format~\cite{cui2022deep}.
This enables us to generate samples from the target distribution with far fewer function evaluations compared to the \texttt{t-walk} sampling algorithm~\cite{christen2010general}.
Tensor-train methods require a predefined grid and a ``normalisation constant'' so that function outputs are within computer precision, but once defined, they reduce the function evaluations per independent sample significantly.
Another advantage of the tensor-train format is that marginal distributions, useful for characterisation of integrals via quadrature, can be calculated at a low computational cost, without any sampling.
To further improve tensor-train methods, we suggest future work should focus on lowering tensor ranks, calculating "normalisation constants" to avoid numerical issues and reducing correlation structures between parameters automatically, all of which we currently have to do by exploratory analysis.
Additionally, choosing accurate interpolation schemes between grid points may be crucial to improving the effectiveness of the approximation.

Our results show that a hierarchical Bayesian approach, which quantifies posterior mean and variance of the parameter (ozone), provides more information than a regularisation approach at comparable computational time.
In regions where the signal strength is low and the data is noise-dominated, we can not recover ozone structures from the ground truth.
When including pressure and temperature describing hyper-parameters within our hierarchical Bayesian model, we find a strong correlation between ozone and pressure, whereas the model and data are uninformative about temperature.
For future work, we recommend developing a more physically informed parametrised model for ozone within the atmosphere, incorporating atmospheric chemistry and other important processes.